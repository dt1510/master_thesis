
\chapter{Introduction}

\section{Motivation and Objectives}
Human and artificial agents interact with the environment. Learning the mechanism of the environment from the observations enables an agent to make prediction based on which one can make rational decisions. The abstraction of the complex interaction between an environment and an agent with its objectives gives a rise to learning problems of different forms.

\subsection{Utility maximization}
\subsection{Knowledge compression}
\subsection{Model learning}
\subsection{Verbatim memorization}
\subsection{Abstraction}
\subsubsection{Pattern Abstraction}
\subsubsection{Concept Abstraction}
\subsection{Information Search}
\subsubsection{Information Confirmation}
\subsubsection{Information Refutation}

learn most probable hypothesis vs learning the most useful hypothesis, from the observations we would like to learn what is the most relevant to us with respect to the circumstances.
ideas:
maximizing what you know - knowledge compression
learning the rules of a game
learning the postulates of the environment
trying the represent the model of the world most succintly
producing a Turing machine with an equivalent code
the ability to survive relies on the prediction
learning knowledge is a manifestation of a human curiosity
our decisions (and an exercise of a free will) can be made better if we understand their impact
\subsection{}

\section{Contributions}

Contributions here.


\section{Statement of Originality}

Statement here.