
\chapter{Introduction}

%\section{Machine learning}

\section{Inductive inference}
Aristotle recognized 2 forms of logical reasoning: a deduction and an induction\cite{sep-aristotle-logic} (3rd form being an abduction recognized later by the philosopher C. S. Peirce(1839-1914)\cite{kimber2012learning}\cite{peirce1935collected}.

A deduction is a sound process of deriving truth from the known facts. Let p be the proposition "All men are mortal." and q be the proposition "Aristotle is a man.". Then we can \emph{deduce} the proposition r "Aristotle is mortal." from p and q.

Given the observations q, r one may \emph{induce} the proposition q as a possible explanation of r from q. However, this process of induction is not sound as in general q, r do not imply p. Indeed, one may induce an explanation p2 "Every being whose name starts with A is mortal." for the observations q, r. But we know that Aphrodite, being a Greek mythology goddess, is not mortal. While  deduced propositions restate the known, an induced proposition provides an explanation for the known and unknown.

The problem of finding a sound explanation for the observations known as a problem of induction has been investigated by Hume\cite{hume1902enquiries}\cite{selby1888treatise}, Popper\cite{keuth2013karl}, Stove\cite{stove1986rationality} and many others.

\section{Inductive logic programming}\cite{muggleton1995inverse}\cite{nienhuys1997foundations}
Inductive logic programming is a subdiscipline of machine learning and logic programming. The goal of inductive logic programming is to study how the theory of inductive inference and logic programming can be used to design theoretical frameworks and programs solving the machine learning problems.

An ILP system is a program that takes as an input a set logical statements $B$ called background knowledge, a set of logical statements $E$ called examples or observations. The output of an ILP system is a set of logical statements $H$ called a hypothesis explaining the observations from the background knowledge. Meaning $E$ can be deduced from $B$ and $H$.

ILP systems have been successfully applied in solving scientific problems with their applications in the areas of mesh design, mutagenicity, river water quality 
\cite{bratko1995applications}. However, the general setting of ILP does not limit their application to these domains, but leaves the space for solving even the hardest problems in the field of machine learning and AI.
\section{Motivation and Objectives}
We explain the value of the classification of ILP systems, summarise the classification conducted in the field of ILP so far and conclude with the objectives of our classication of ILP systems.

\subsection{Value of classification of ILP systems}
Many systems and theoretical frameworks have been designed that use the theory of inductive inference to find a hypothesis $H$ for observations $E$ from the background knowledge $B$. It is important to compare these systems and evaluate their level of contribution to solving the issues in ILP in order to identify and understand the most successful aproaches in order to create new methods by their combination and to direct the research in ILP accordingly. 

The most important benefits of the classification of ILP systems are:
\begin{itemize}
\item Identification of the successful methods,
\item Identification of the points for improvement,
\item Identification of the new emerging techniques.

\item Understanding of the differences between various approaches,
\item Selecting an ILP system for a specific application

\item Formalisation and standarization of common concepts,
\end{itemize}

\subsection{Current classification of ILP systems}
To the author's knowledge no extensive experimental or theoretical comparisons have been made (cf. \fullref{ch:evaluation}). ILP researchers typically evaluate their new designed system empirically based on the success of finding a hypothesis for a carefully chosen datasets that may not be representative of dataset in the real world.

\subsection{Objectives}

\section{Summary of thesis achievements}

\section{Project development}
In order to classify ILP systems comprehensively, the author first seeked to learn the foundations of ILP from which one could build up a systematic way of comparison of ILP systems as mathematical objects. As the author to his great dissappointment realized that there were no foundations at the theoretical level required, the author decided to pursue his own development of the foundations. During the development, the author discovered that each ILP system solves a different ILP task (as a result of ILP systems not designed on the common foundations) and therefore ILP systems are incomparable on the majority of the properties characterising an ILP system as a mathematical object. Hence the author derived a definition of an ILP task unifying a small class of the properties of an ILP system and retreated to the experimental comparison of ILP systems. The author discoreved that ILP systems do not function as specified by their theoretical frameworks. Hence, in the end, the author pursued a development of the ideas on the theoretical frameworks based on the newest existing literature in hope one would be able to move the classification further at the theoretical level. During this stage the author extended the newest theoretical results and implemented an ILP system encompassing them. The author defined new concepts that were used previously intuitively or at an implementation level only which enabled him to extend the classification of ILP systems to the final state.

\section{Overview}
The rest of the thesis is organized as follows:
\begin{itemize}
\item \subsubsection{\fullref{ch:background}} We provide a common background for languages, logics, model theory and logic programming.
\item \subsubsection{\fullref{ch:inductive_logic_programming}}
We provide background for inductive logic programming and list the ILP systems to be classified.
\item \subsubsection{\fullref{ch:bias}}
We formalise the theory of a language bias and classify ILP systems by their language and search bias.
\item \subsubsection{\fullref{chap:classification_of_ilp_systems}}
We perform experimental classification of ILP systems based on the classes of problems they can solve and on the robustness of their implementations.
\item \subsubsection{\fullref{inverse_subsumption_for_complete_explanatory_induction}}
We present a complete inverse subsumption with minimal complements algorithm by Yamamoto et al. \cite{yamamoto2012inverse} for finding a hypothesis with anti-subsumption. Further, we extend this algorithm to first-order theories, negative examples and relax it while preserving the completeness. We introduce two inverse subsumption operators and prove their completeness. The chapter is concluded with the proof of Imparo's completeness by inverse subsumption.
\item \subsubsection{\fullref{chap:rationale_ilp_system}}
We summarise our prototype ILP system implementation Rationale based on the theoretical extensions from the preceeding chapter.
\item \subsubsection{\fullref{ch:evaluation}}
We evaluate the achievements of the thesis and their relative contribution wrt what has been done in the field of ILP.
\item \subsubsection{\fullref{ch:conclusions}}
We conclude the chapter summarising the new knowledge that the thesis created and point out the future directions for the work.
\item \subsubsection{Appendix}
Appendix contains around 100 experimental examples of learning problems with explanations of the results for each ILP system we classified:\\
Progol \ref{appendix_progol},
Aleph \ref{appendix_aleph},
Toplog \ref{appendix_toplog},  
Xhail \ref{appendix_xhail},
Imparo \ref{appendix_imparo},
Tal \ref{appendix_tal}.
\end{itemize}