\chapter{Background}\label{ch:background}

\section{Prerequisites}
We assume the reader is familiar with basic concepts in several areas and consider the following references to be potentially useful:
\begin{enumerate}
\item Foundations of Inductive Logic Programming by Cheng and Wolf \cite{nienhuys1997foundations}
\item Model theory: An introduction by David Marker \cite{marker2002model}
\end{enumerate}
However, the usage of the concepts does not exceed its rudimentary application and the reader is encouraged to proceed further even if one may not be familiar with all the areas outlined, we suggest to use the literature when the need arises.

We provide a brief summary of concepts from a logic language, model theory, logics and logic programming.
\section{Logic language}
A logic language is a specification for representing logical statements in a logic program (an input to an ILP system) and in mathematical proofs.

\begin{defn}
An \emph{alphabet} is a set $\mathcal{A}$ of elements called \emph{letters} of an alphabet.
\end{defn}

\begin{defn}
A word over an alphabet $\mathcal{A}$ is a finite sequence (a string) of letters of the alphabet $\mathcal{A}$.
\end{defn}

\begin{defn}
A formal language $L$ is a set of words over the specified alphabet $\mathcal{A}$, i.e. $L \subseteq \powerset{\mathcal{A}}$.
\end{defn}

\begin{exmp}
Let $R$ be a regular expression $0001*$.
\begin{itemize}
\item The alphabet is the set $\mathcal{A}={\epsilon, 0,1,*}$. $0$, $1$, $*$ are called letters of $\mathcal{A}$.
\item The words over $\mathcal{A}$ are $000$.
\end{itemize}
\end{exmp}

Formal languages we are going to use are first-order languages:
\begin{defn}
A \emph{language} $L$ is a formal language given by the grammar of first-order logic and the additional data $\mathcal{L}$ called the signature of $L$:
\begin{enumerate}
\item a set $\mathcal{C}$ of letters called constant symbols,
\item a set $\mathcal{R}$ of letters called relation symbols and a positive integer (an arity) $n_R$ for each $R \in \mathcal{R}$,
\item a set $\mathcal{F}$ of letters called function symbols and a positive integer (an arity) $n_f$ for each $f \in \mathcal{F}$.
\end{enumerate}
\end{defn}

\begin{remark}
We will often talk about the language $L$ by referring to its signature $\mathcal{L}$. This should not cause a confusion as the signature $\mathcal{L}$ and the grammar of the first order logic uniquely determine the language $L$.
\end{remark}

\begin{exmp}
The language $\mathcal{L}_{EST}$ of elementary set theory is given by one constant symbol called an empty set, $\mathcal{C}=\{\emptyset\}$, one relation symbol called a set membership $\mathcal{R}=\{\in\}$ with a positive integer (arity) $n_{\in}=2$. The set of function symbols $\mathcal{F}$ are the set operations of a union, an intersection, each with an arity $2$.
\end{exmp}

\begin{defn}
Given a language $L$, $\phi$ is an $L$-formula iff $\phi \in L$.
\end{defn}

\begin{defn}
Let $\Sigma \subseteq L$, $\phi \in L$, then we say that $\Sigma$ \emph{equals} $\phi$ iff $\phi$ is a conjunction of the formulas $\psi \in \Sigma$, i.e. $\land \Sigma = \phi$. We abuse the symbol $=$ to denote the defined relation, writing $\Sigma=\phi$.
\end{defn}

\begin{remark}
If $\Sigma \subseteq L$, $\phi \in L$, $\Sigma=\phi$, then under the defined equality relation it is allowed to write both, $\Sigma \subseteq L$ and $\Sigma \in L$, for the later meaning $\Sigma = \phi \in L$.
\end{remark}

\section{Model theory\cite{marker2002model}}
We will reason in first-order model theory using the logic language we defined and will use the first-order definitions from Model Theory by Marker\cite{marker2002model}.

\begin{defn}
An \emph{$\mathcal{L}$-structure} (\emph{a model}) $\mathcal{M}$ is given by the following data:
\begin{enumerate}
\item a nonempty set $M$ called the universe, domain, or underlying set of $\mathcal{M}$;
\item a function $f^{\mathcal{M}} : M^{n_f} \to M$ for each $f \in \mathcal{F}$;
\item a set $\mathcal{R}^{\mathcal{M}} \subseteq M^{n_R}$ for each $R \in \mathcal{R}$;
\item an element $c^\mathcal{M} \in M$ for each $c \in C$.
\end{enumerate}
We refer to $f^\mathcal{M}, R^\mathcal{M}, c^\mathcal{M}$ as the interpretations of the symbols $f ,R, c$.
\end{defn}

\begin{defn}(\emph{Interpretation}).
Let $\phi$ be a formula with free variables from
$\overline{v}=(v_{i_1},...,v_{i_m})$,
and let
$\overline{a}=(a_{i_1}, ..., a_{i_m})) \in M^m$.
We inductively define $\mathcal{M} \models \phi(\overline{a})$
as follows.\\
i) If $\phi$ is $t_1=t_2$, then $\mathcal{M} \models \phi(\overline{a})$ if $t_1^\mathcal{M}(\overline{a})=t_2^\mathcal{M}(\overline{a})$.\\
ii) If $\phi$ is $R(t_1, ..., t_{n_R})$, then $\mathcal{M} \models \phi(\overline{a})$ if $(t_1^\mathcal{M}(\overline{a}), ..., t_{n_R}^\mathcal{M}(\overline{a})) \in R^\mathcal{M}$.\\
iii) If $\phi$ is $\neg \psi$, then $\mathcal{M} \models \phi(\overline{a})$ if $\mathcal{M} \not\models \psi(\overline{a})$.\\
iv) If $\phi$ is $(\psi \land \theta)$, then $\mathcal{M} \models \phi(\overline{a})$ if $\mathcal{M} \models \psi(\overline{a})$ and
$\mathcal{M} \models \theta(\overline{a})$.\\
v) If $\phi$ is $(\psi \lor \theta)$, then $\mathcal{M} \models \phi(\overline{a})$ if $\mathcal{M} \models \psi(\overline{a})$ or $\mathcal{M} \models \theta(\overline{a})$.\\
vi) If $\phi$ is $\exists v_j \psi(\overline{v},v_j)$, then $\mathcal{M} \models \phi(\overline{a})$ if there is $b \in M$ such that $\mathcal{M} \models \psi(\overline{a},b)$.\\
vii) If $\phi$ is $\forall v_j \phi(\overline{v}, v_j)$, then $\mathcal{M} \models \phi(\overline{a})$ if $\mathcal{M} \models \phi(\overline{a}, b)$ for all $b \in M$.\\
If $\mathcal{M} \models \phi(\overline{a})$ we say that $\mathcal{M}$ \emph{satisfies} $\phi(\overline{a})$ or $\phi(\overline{a})$ is \emph{true} in $\mathcal{M}$.
\end{defn}

\begin{exmp}
$\mathcal{M}_{India}$ is a model given by
$M=\{milk^\mathcal{M}, curry^\mathcal{M}\}$,
$R=\{TastesHot, IsWhite\}$,$\mathcal{F}=\emptyset$,
$TastesHot=\{milk^\mathcal{M}, curry^\mathcal{M}\}$,
$IsWhite=\{milk^\mathcal{M}\}$ with a canonical mapping of constants from
 $C=\{milk, curry\}$ to $M$. Therefore $\mathcal{M}$ is a model of a formula
$\phi=TastesHot(milk) \wedge \neg IsWhite(curry)$ denoting by
$\mathcal{M} \models \phi$ reading ``a model $\mathcal{M}$ entails a formula $\phi$".
\end{exmp}

\begin{remark}
An $L$-structure is an $\mathcal{L}$-structure iff $\mathcal{L}$ is a signature of a language $L$.
\end{remark}

\begin{defn}
Given a set $T$ of logic sentences and a model $\mathcal{M}$. If $\forall \phi \in T. \mathcal{M} \models \phi$, then $\mathcal{M}$ is \emph{a model of $T$} and $T$ is \emph{a theory of $\mathcal{M}$}.
\end{defn}

\begin{defn}
$A$ is an axiomatization of the theory $T$ iff $M(A)=M(T)$.
\end{defn}

\begin{remark}
Typically we put further restrictions on the properties of the axiomatization $A$ that do not hold for $T$. E.g. $A$ must have finitely many axioms for $T$ with infinitely many axioms.
\end{remark}

\begin{defn}
$A$ is an axiomatization of the theory $O$ from the theory $B$ iff $M(A \union B)=M(O)$.
\end{defn}

\section{Logics\cite{sep-logic-nonmonotonic}}
ILP systems may use different types of logics, their semantics is induced by a consequence operator.
We adapt definitions of consequence operators from \cite{sergot2005knowledgeRepresentationLectureNotes}.

\begin{defn}
A \emph{consequence operator} on a set $L$ is a function $Cn:\powerset{L} \to \powerset{L}$ satisfying the following conditions for all sets $X, Y \subseteq L$:\\
1) $X \subseteq Cn(X)$ inclusion,\\
2) $Cn(Cn(X)) \subseteq Cn(X)$ closure.\\
\end{defn}

\begin{defn}
A consequence operator $Cl:\mathcal{L} \to \mathcal{L}$ is a \emph{classical (or monotone) consequence operator} iff it satisfies \emph{monotony}:
$X \subseteq Y \implies Cl(X) \subseteq Cl(Y)$.
\end{defn}

\begin{defn}
A consequence operator $Cn:\mathcal{L} \to \mathcal{L}$ is a \emph{non-monotone consequence operator} iff it does not satisfy monotony, i.e.
$\exists X, Y \subseteq L, X \subseteq Y \land Cn(X) \not\subseteq Cn(Y)$.
\end{defn}

\begin{defn}
A logic is \emph{monotonic} iff its consequence operator is monotone.
A logic is \emph{non-monotonic} iff its consequence operator is non-monotone.
\end{defn}

\begin{exmp}
First-order logic is monotonic.
\end{exmp}

Logics used in ILP are usually non-monotonic.

\section{Logic Programming}
Inductive logic programming (ILP) uses the theory of logic programming for the representation of its input and output theories and for its proof procedures.
The following definitions and concepts are adaptations from Dianhuan Lin's Master thesis\cite{lin2009efficient}.

\subsection{Basic concepts and notation of Logic Programming\cite{lin2009efficient}}

\begin{defn}
A term is a constant, variable, or the application of a function symbol to the appropriate number of terms. A ground term is a term not containing variables.
An atom is the application of a predicate symbol to the appropriate number of terms. A literal is an atom or the negation of an atom.
\end{defn}

\begin{defn}
A definite goal is a clause of the form
$\leftObjectImplies B_1 , ..., B_n$.
where $n > 0$ and each $B_i$ is an atom.
Each $B_i$ is called a subgoal of the goal.
\end{defn}

\begin{defn}
A definite clause is a clause of the form
$A \leftObjectImplies B_1 , ..., B_n$
which contains precisely one positive literal $A$.
$A$ is called the head and $B_1$ , ..., $B_n$ is called the body of the clause.
A Horn clause is either a definite clause or a definite goal.
A unit clause consists of a single atom.
\end{defn}

\begin{remark}
When using set operations and relations on clauses, we will think of a clause as a set of literals whose conjunction is logically equivalent to it.
\end{remark}

\begin{defn}
A logic program is a set of clauses representing their conjunction.
\end{defn}
We will reason about finite logic programs only.

\subsection{SLD-resolution\cite{lin2009efficient}}
\begin{defn}
A \emph{substitution} $\theta$ is a finite set of the form
$\{v_1\/t_1 , .., v_n \/t_n \}$,
where each $v_i$ is a variable, each $t_i$ is a
term distinct from $v_i$ and the variables $v_1 , .., v_n$ are distinct. Each element $v_i \/t_i$ is called a binding
for $v_i$. $\theta$ is called a ground substitution if the $t_i$ are all ground terms.
\end{defn}

\begin{defn}
An expression is either a term, a literal, or a conjunction or disjunction of literals.
A simple expression is a term or a literal.
\end{defn}

\begin{defn}
\emph{Unification.}
Let $\Sigma$ be a finite set of expressions. A substitution $\theta$ is called a unifier for $\Sigma$ if $\Sigma \theta$ is a singleton (a
set containing exactly one element). A unifier $\theta$ for $\Sigma$ is called a most general unifier (mgu) for $\Sigma$
if, for each unifier $\theta$ of $\Sigma$ , there exists a substitution $\Gamma$ such that $\theta = \Sigma \Gamma$
\end{defn}
Details about unification algorithm can be found in \cite{lloyd1987foundations}.

\begin{defn}
Let $G$ be $\leftObjectImplies A_1 , ..., A_m , ..., A_k$ and
$C$ be $A \leftObjectImplies B_1 , ..., B_q$.
Then $G'$ is derived from $G$ and
$C$ using mgu $\theta$ if the following conditions hold:
i) $A_m$ is an atom, called the selected atom, in $G$.
ii) $\theta$ is an mgu of $A_m$ and $A$.
iii) $G'$ is the goal $\leftObjectImplies (A_1 , ..., A_{m−1} , B_1, ..., B_q , A_{m+1} , ..., A_k )\theta$, $G'$ is
called a resolvent of $G$ and $C$.
\end{defn}
SLD-resolution stands for SL-resolution for Definite clauses. SL stands for Linear resolution with Selected function.

\begin{defn}
\emph{SLD-derivation}.
Let $\Sigma$ be a set of clauses and $C$ a clause. A derivation of $C$ from $\Sigma$ is a finite sequence of clauses
$R_1 , .., R_k = C$, such that each $R_i$ is either in $\Sigma$, or a resolvent of two clauses in $\{R_1 , ..., R_{i−1} \}$. If
such a derivation exists, $\Sigma \vdash_r C$. Thus $C$ can be derived from $\Sigma$.
\end{defn}

SLD-refutation is a special case of SLD-derivation which derives empty clause $\emptyclause$.

\subsubsection{Resolution\cite{kimber2012learning}}
The resolution inference rule for a literal $\phi$ and clauses $\psi, \chi$ is
$(\phi \vee \psi) \wedge (\neg \phi \wedge \chi) \vdash \phi \vee \chi$.

\subsection{Representation of logic theories}\label{background_representation_of_logic_theories}
A \emph{logic theory} is a subset of a logic language $L$.
\emph{Clausal theories} allow sentences to be clauses, but one may restrict the theories to be \emph{extended logic programs}, \emph{normal logic programs}, 
\emph{Horn theories}, \emph{definite logic programs}, \emph{Datalog programs}, \emph{literals}, \emph{atoms}. For example, Imparo\cite{kimber2012learning} requires background knowledge and a hypothesis to be a definite logic programs and examples $E^{+}$ and $E^{-}$ to be ground atomic whereas Yamamoto et al. reason about the explanatory induction in \cite{yamamoto2012inverse} with any clausal theories $B, E, H$. This has an impact on the complexity of the learning problem as every definite theory can be expressed as a clausal theory, but not vice versa.