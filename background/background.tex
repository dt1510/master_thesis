
\chapter{Background Theory}

\label{ch:background}

\section{Introduction}

\section{Inductive Logic Programming problem}

\subsection{Definitions}

A language of enquiry $\mathcal{L}$ is a model-theoretic language given by constant symbols, relation symbols and function symbols. A language of enquiry $\mathcal{L}$ is a superlanguage of two languages: an observational language $\mathcal{L}_o$ and a hypothesis language $\mathcal{L}_h$. We call the computable restrictions on the language $\mathcal{L}$ to the hypothesis language $\mathcal{L}_h$ to be a language bias.

Every language of enquiry has exactly one $\mathcal{L}$-structure (model) $\mathcal{M}$ that is called the reality of $\mathcal{L}$. The class of all $\mathcal{L}$-structures are called the hypothetical models that are models of $\mathcal{L}$.

Given a set of $\mathcal{L}$-formulas $\Sigma$, we denote $M(\Sigma)$ to be the class of hypothetical models of $\Sigma$. The set of observations $O$ is the set of the $\mathcal{L}_o$-formulas.
Given a set of observations $O$, we say that any $\mathcal{L}$-formula $Q$ divides the models of $O$ into models of $Q \wedge O$ denoted $M(O \union {Q})$ and the models of $\neg Q \wedge O$ denoted $M(O \union {Q})$.
The background knowledge (or current induced theory) $B$ is a set of formulas in a hypothesis language $\mathcal{L}_h$.

\subsection{Ultimate problem of scientific discovery}
Statement: Given a language of enquiry $\mathcal{L}$, find the reality $\mathcal{M}$.

We do not consider the situation in which we may desire to learn approximate models of the reality. In general, depending on our language of enquiry, the reality may not exist, however our definition considers only the languages of enquiry with exactly one reality. Some realities, e.g. (regular expression) are learnable, others may be not. The ultimate problem consists of numerous incremental problems.

\subsection{Incremental problem of scientific discovery}
We would like to find the reality most efficiently, where the efficiency may be measured by some criteria - e.g. time and resources. The following simplified definition does not take into the account that some questions may be found more efficiently than others.

Set a probability measure $\mu:M(\emptyset) \to [0,1] \subset \mathcal{R}$ over hypothetical models of a language of enquiry $\mathcal{L}$. Given a background knowledge $B$, find a question $Q$ that minimizes the following equation: $|M(B \union {Q})-M(B \union {\neg Q})|$, i.e. a question that divides the models of $B$ most evenly.

\subsection{Definition of a problem of an induction}
Fix a language of enquiry $\mathcal{L}$ with its sublanguages $\mathcal{L}_o$, $\mathcal{L}_h$. Define an equivalence relation over the sets of the
 $\mathcal{L}$-formulas: $\Sigma \sim \Gamma \iff K(\Sigma)=K(\Gamma)$
where $K$ is the Kolmogorov complexity with respect to the description language $\mathcal{L}$. Given background knowledge $B$, observations $O$, find a representative of the axiomatization with the least Kolmogorov complexity of the theory $\Sigma$ whose elementary class of models are precisely the models of $B \union O$. We call the representative $\Sigma$ an induced theory from the background knowledge $B$ and the observations $O$.

Note: A definition concerned with the non-monotonic logics will need to be adapted.

\subsection{Definition of a scientific method}
A scientific method is an algorithm used to solve the ultimate problem of the scientific discovery:

0. Start with the emty theory $B_i=\emptyset$, set $i=0$.

1. If $B_i$ has the only model, then terminate, $B_i$ is a complete axiomatization of the reality of the language of enquiry.

2. Solve an incremental problem of scientific discovery by finding a question $Q$ given the current induced theory $B_i$.

3. make an observation - ask an environment oracle a question $Q$.

4. induce a more precise theory $B_{i+1}:=B \union \{Q\}$ if the oracle says that $Q$ is true, $B_{i+1}:=B \union \{\neg Q\}$ if the oracle says that $\neg Q$ is true.

5. Increment $i$ and start from step 1 again.

\subsection{Definition of an ILP problem}

\subsection{Definition of an ILP system}