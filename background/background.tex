
\chapter{Background Theory}

\label{ch:background}

\section{Introduction}

\section{Inductive Logic Programming problem}

An ILP Problem is as follows: given a set of examples, what is the hypothesis likely to have produced them?

\subsection{ILP: Normal problem setting}
Cheng and Wolf define the ILP problem in the following way:

Given: A finite set of clauses $B$ (background knowledge), and sets of clauses $E+$ and $E-$ (positive and negative examples).
Find: A theory $E$, such that $E \union B$ is correct with respect to $E+$ and $E-$.

\subsection{ILP: The Nonmonotonic problem setting}
From Cheng and Wolf:

Given: Two sets $Z+$ and $Z-$ of Herbrand interpretations (positive and negative examples).
Find: A theory $P$, which is true under each $I E Z+$ and false
under each $I E Z-$.

\subsection{Considtions for constructing an hypothesis}
Necessity: B |= E +
Sufficiency: B ∧ h |= E +
Weak consistency: B ∧ h |= box
Strong consistency: B ∧ h ∧ E − |= box

\subsetction{Model theoretic definition}
ILP problem is 

\subsection{Definition of an ILP problem}
Fix a formal system with a language $L$. Given a set of true formulas $E+$ and false formulas $E-$ find a theory $\Sigma$ satisfying $E+$ and consistent with $E-$ whose elementary class of models would be the largest possible.