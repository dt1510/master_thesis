\chapter{Appendix - evaluation of Toplog on examples}

\section{Toplog}
\subsection{Mode declarations order bias}
Consider the Toplog program:

\begin{minipage}[t]{.35\textwidth}
\begin{lstlisting}
:-modeh(r(+type)).
:-modeb(1, p1(+type)).
:-modeb(1, p2(+type)).
:-set(maximum_literals_in_hypothesis, 10).
\end{lstlisting}
\end{minipage}
\begin{minipage}[t]{.20\textwidth}
\begin{lstlisting}
p1(a1).
p1(a2).
p1(a3).
\end{lstlisting}
\end{minipage}
\begin{minipage}[t]{.20\textwidth}
\begin{lstlisting}
p2(a1).
p2(a2).
p2(a3).
\end{lstlisting}
\end{minipage}
\begin{minipage}[t]{.25\textwidth}
\begin{lstlisting}
example(r(a1),1).
example(r(a2),1).
example(r(a3),1).
\end{lstlisting}
\end{minipage}


Then Toplog learns a hypothesis $r(A) :- p1(A).$ however if we change the mode declaration to the following:
\begin{lstlisting}
:-modeh(r(+type)).
:-modeb(1, p2(+type)).
:-modeb(1, p1(+type)).
\end{lstlisting}
then Toplog learns a hypothesis $r(A) :- p2(A).$ instead. This experiment therefore shows that Toplog's bias depends on the order of the mode declaration statements.

\subsection{Learning inconsistent hypothesis}
Toplog can learn inconsistent hypotheses:

\begin{minipage}[t]{.40\textwidth}
\begin{lstlisting}
:-modeh(woman(+person)).
:-modeb(1, student(+person)).
\end{lstlisting}
\end{minipage}
\begin{minipage}[t]{.20\textwidth}
\begin{lstlisting}
student(ann).
student(tom).
\end{lstlisting}
\end{minipage}
\begin{minipage}[t]{.20\textwidth}
\begin{lstlisting}
example(woman(ann),9).
example(woman(tom),-1).
\end{lstlisting}
\end{minipage}


gives an output hypothesis $woman(A) :- student(A)$ although $tom$ is not a woman as indicated by a negative example $example(woman(tom),-1)$.
But this is intentional, furthermore Toplog provides the information that the default accuracy of its hypothesis is 90\% as $example(woman(ann,9)$ corresponds to 9 positive examples correctly classified out of 10.

\subsection{One head predicate bias}
All the clauses in the hypotheses space of Toplog have to have the same predicate symbol in its head.

\begin{minipage}[t]{.40\textwidth}
\begin{lstlisting}
:-modeh(woman(+person)).
:-modeb(1, female(+person)).
:-modeb(1, male(+person)).
\end{lstlisting}
\end{minipage}
\begin{minipage}[t]{.20\textwidth}
\begin{lstlisting}
male(tom).
female(ann).
\end{lstlisting}
\end{minipage}
\begin{minipage}[t]{.20\textwidth}
\begin{lstlisting}
example(woman(ann),9).
example(woman(tom),-1).
\end{lstlisting}
\end{minipage}

produces a hypothesis \tc{woman(A) :- female(A)} but adding a second
head mode declaration \\
\tc{:-modeh(man(+person)).} after \tc{:-modeh(woman(+person)).} overwrites the first one.

\begin{minipage}[t]{.40\textwidth}
\begin{lstlisting}
:-modeh(woman(+person)).
:-modeh(man(+person)).
:-modeb(1, female(+person)).
:-modeb(1, male(+person)).
\end{lstlisting}
\end{minipage}
\begin{minipage}[t]{.20\textwidth}
\begin{lstlisting}
male(tom).
female(ann).
\end{lstlisting}
\end{minipage}
\begin{minipage}[t]{.20\textwidth}
\begin{lstlisting}
example(woman(ann),9).
example(woman(tom),-1).
\end{lstlisting}
\end{minipage}

produces no hypotheses.

\subsection{Unlearnability of the term structure}
Consider one would like to learn a concept of a Kleene star operator on the terms evident from the following examples:
\begin{lstlisting}
%target concept: in_language(s(A)) :- in_language(A).
:-modeh(in_language(s(+word))).
:-modeb(1, in_language(+word)).

example(in_language(s(epsilon)),1).
example(in_language(s(s(epsilon))),1).
example(in_language(s(s(s(epsilon)))),1).
example(in_language(s(s(s(s(epsilon))))),1).
\end{lstlisting}

However, this is not learnable since Toplog's mode declarations do not allow to specify\\
\tc{in\_language(s(A))} in a head neither in a body. Toplog outputs:
"Couldn't start model. (no problem defined?)".

\subsection{Clausal bias}
Since Toplog's hypotheses space consists of clauses, it cannot learn two concepts at one time.

\begin{lstlisting}
:-modeh(object(+type)).
:-modeb(10, blue(+type)).
:-modeb(10, green(+type)).
:-set(maximum_literals_in_hypothesis, 10).
\end{lstlisting}

\begin{minipage}[t]{.35\textwidth}
\begin{lstlisting}
blue(ball).
blue(pen).
blue(bag).
green(shirt).
green(grass).
green(tree).\end{lstlisting}
\end{minipage}
\begin{minipage}[t]{.20\textwidth}
\begin{lstlisting}
example(object(ball),1).
example(object(pen),1).
example(object(bag),1).
example(object(shirt),1).
example(object(grass),1).
example(object(tree),1).
\end{lstlisting}
\end{minipage}

will produce a hypothesis \tc{object(A) :- blue(A)} since 
\tc{:-modeb(10, blue(+type)).} was defined first instead of learning both
concepts in a hypothesis:
\tc{object(A) :- blue(A); green(A)} where the \tc{;} represents a disjunction.

\subsection{Inability to make deductions from observations}
Toplog does not learn any hypothesis from the following program:

\begin{lstlisting}
:-modeh(t(+type)).
:-modeb(1, p(+type)).
:-set(maximum_literals_in_hypothesis, 10).
\end{lstlisting}
\begin{minipage}[t]{.40\textwidth}
\begin{lstlisting}
p(a1).
p(a2).
p(a3).
t(A) :- r(A).
\end{lstlisting}
\end{minipage}
\begin{minipage}[t]{.40\textwidth}
\begin{lstlisting}
example(r(a1),1).
example(r(a2),1).
example(r(a3),1).
\end{lstlisting}
\end{minipage}

Although from the positive examples and the background knowledge included one may deduce examples:
\begin{lstlisting}
example(t(a1),1).
example(t(a2),1).
example(t(a3),1).
\end{lstlisting}
which if included in the Toplog program, then a hypothesis
\tc{tc(A) :- p(A)} would be learnt.
Therefore the application of the background knowledge to the observations in Toplog is limited.

Nevertheless, Toplog can still learn by making deductions from its background knowledge.

\begin{lstlisting}
:-modeh(t(+type)).
:-modeb(1, r(+type)).
:-set(maximum_literals_in_hypothesis, 10).
\end{lstlisting}
\begin{minipage}[t]{.35\textwidth}
\begin{lstlisting}
p(a1).
p(a2).
p(a3).
r(A) :- p(A).
\end{lstlisting}
\end{minipage}
\begin{minipage}[t]{.20\textwidth}
\begin{lstlisting}
example(t(a1),1).
example(t(a2),1).
example(t(a3),1).
\end{lstlisting}
\end{minipage}

would give a hypothesis \tc{t(A) :- r(A).}

\subsection{Predicate generalization impossible}
Toplog cannot generalize over the predicates as this cannot be expressed in the mode declarations bias, consider the illustrative example.
\begin{lstlisting}
example(swims(magician),1).
example(flies(magician),1).
example(cooks(magician),1).
example(makes_fire(magician),1).
\end{lstlisting}
The hypothesis $\forall x. x(magician)$ cannot be induced. Depending on our language and the predicates it contains and the context of the problem, we may need to express hypotheses as second-order logic formulas.

\subsection{Ground atom example}
Toplog has a limitation that every example can be only a ground atom.
