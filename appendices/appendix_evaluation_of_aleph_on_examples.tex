\chapter{Appendix - evaluation of Aleph on examples}\label{appendix_aleph}

\section{Aleph's capabilities}

\subsection{Multi-clausal learning}\label{aleph_multiclausal_learning}
Aleph can learn several one-clausal hypotheses at one time.

\begin{minipage}[t]{.45\textwidth}
\begin{lstlisting}
:-modeh(*, woman(+person)).
:-modeh(*, man(+person)).
:-modeb(*, male(+person)).
:-modeb(*, female(+person)).
\end{lstlisting}
\end{minipage}
\begin{minipage}[t]{.20\textwidth}
\begin{lstlisting}
:- determination(man/1,male/1).
:- determination(woman/1,female/1).
\end{lstlisting}
\end{minipage}

\begin{minipage}[t]{.30\textwidth}
\begin{lstlisting}
%background theory
male(bob).
male(tom).

female(ann).
female(mary).
\end{lstlisting}
\end{minipage}
\begin{minipage}[t]{.30\textwidth}
\begin{lstlisting}
%positive examples
woman(ann).
woman(mary).
man(bob).
man(tom).
\end{lstlisting}
\end{minipage}
\begin{minipage}[t]{.30\textwidth}
\begin{lstlisting}
%negative examples
man(ann).
man(mary).
woman(bob).
woman(tom).
\end{lstlisting}
\end{minipage}

returns hypotheses

\begin{lstlisting}
woman(A) :- female(A).
man(A) :- male(A).
\end{lstlisting}

Note, this is possible in Imparo, but not in Toplog.

\subsection{Learnability of predicates of greater arity}
Aleph can induce hypotheses that contain predicates of arity greater than 1.
\begin{lstlisting}
%background theory
:-modeh(*, english(+person)).
:-modeb(*, english_couple(+person, -person)).
:-modeb(*, english(+person)).

:- determination(english/1, english_couple/2).
\end{lstlisting}
\begin{minipage}[t]{.40\textwidth}
\begin{lstlisting}
english_couple(adam, alice).
english_couple(bob, barbara).
english_couple(jack, jane).
\end{lstlisting}
\end{minipage}
\begin{minipage}[t]{.27\textwidth}
\begin{lstlisting}
%positive examples
english(adam).
english(jack).
\end{lstlisting}
\end{minipage}
\begin{minipage}[t]{.25\textwidth}
\begin{lstlisting}
%negative examples
english(budha).
\end{lstlisting}
\end{minipage}

returns back a generalized hypothesis
\begin{lstlisting}
english(A) :- english_couple(A,B).
\end{lstlisting}.

Note, that since the induce hypothesis does not use the variable \tc{B}, the variable had to be specified in a mode declaration with a minus sign as \tc{-person}.

Similarly, learning of hypotheses with polyadic predicate symbols in their head is possible.

\begin{lstlisting}
:-modeh(*, english_couple(+person, +person)).
:-modeb(*, english(+person)).
:- determination(english_couple/2, english/1).
\end{lstlisting}

\begin{minipage}[t]{.50\textwidth}
\begin{lstlisting}
%background theory
english_couple(adam, alice).
english_couple(bob, barbara).
english(alice).
english(jane).
english(barbara).
english(adam).
english(bob).
english(jack).
\end{lstlisting}
\end{minipage}
\begin{minipage}[t]{.20\textwidth}
\begin{lstlisting}
%positive examples
english_couple(adam, alice).
english_couple(bob, barbara).
english_couple(jack, jane).

%negative examples
english_couple(budha, karma).
english_couple(jack, shreedipta).
english_couple(amir, jane).
\end{lstlisting}
\end{minipage}

returns a hypothesis

\begin{lstlisting}
english_couple(A,B) :- english(B), english(A).
\end{lstlisting} which has in its head a predicate of arity 2.

\subsection{Term structure learnability}
Aleph can learn the hypotheses whose terms contain function symbols.

\begin{minipage}[t]{.50\textwidth}
\begin{lstlisting}
:-modeh(*, woman(sister(+person))).
:-modeb(*, anybody(+person)).

:-determination(woman/1, anybody/1).

anybody(jane).
anybody(susan).
anybody(bob).
\end{lstlisting}
\end{minipage}
\begin{minipage}[t]{.20\textwidth}
\begin{lstlisting}
%positive examples
woman(sister(bob)).
woman(sister(jane)).
woman(sister(susan)).

%negative examples
woman(sister(frog)).
\end{lstlisting}
\end{minipage}

returns a hypothesis
\begin{lstlisting}
woman(sister(A)) :- anybody(A).
\end{lstlisting}
In comparison, Imparo can learn the term structure but Toplog cannot.

\subsection{Generalization downwards}\label{aleph_generalization_downwards}
Aleph can learn the hypotheses of the form $P(s(x)) \objectImplies P(x)$.

\begin{minipage}[t]{.50\textwidth}
\begin{lstlisting}
:-modeh(*, even(+number)).
:-modeb(*, even(+number)).
:-modeb(*, even(s(+number))).
:-modeb(*, even(s(s(+number)))).
:- determination(even/1, even/1).

%background theory
even(s(s(s(s(s(s(s(s(0))))))))).
\end{lstlisting}
\end{minipage}
\begin{minipage}[t]{.20\textwidth}
\begin{lstlisting}
%positive examples
even(0).
even(s(s(0))).
even(s(s(s(s(0))))).
even(s(s(s(s(s(s(0))))))).

%negative examples
even(s(0)).
even(s(s(s(0)))).
\end{lstlisting}
\end{minipage}

produces a hypothesis
\begin{lstlisting}
even(A) :- even(s(s(A))).
\end{lstlisting}
which comes as a surprise since neither Imparo nor Toplog can learn the hypotheses whose variable in head is wrapped by function symbols in its body literals.

\section{Aleph's assumed limitations}

\subsection{Unlearnability of negative examples}
Aleph learns positive examples, but cannot learn the negative ones.

\begin{minipage}[t]{.50\textwidth}
\begin{lstlisting}
:-modeb(*, woman(+person)).
%background theory
%positive examples
\end{lstlisting}
\end{minipage}
\begin{minipage}[t]{.20\textwidth}
\begin{lstlisting}
%negative examples
woman(bob).
woman(tom).
\end{lstlisting}
\end{minipage}

does not return any hypothesis, even if we specify the hypothesis in the mode declaration by \tc{:- woman(+person)}.

\subsection{Determination declaration requirement}\label{subsec:aleph_determination_declaration_requirement}
The hypotheses space cannot be defined with the mode declarations alone, mode declarations have to be supplied with the determination declarations.

\begin{minipage}[t]{.50\textwidth}
\begin{lstlisting}
:-modeh(*, woman(+person)).
:-modeb(*, female(+person)).
%:- determination(woman/1,female/1).

%background theory
female(ann).
female(mary).
\end{lstlisting}
\end{minipage}
\begin{minipage}[t]{.30\textwidth}
\begin{lstlisting}
%positive examples
woman(ann).
woman(mary).

%negative examples
woman(bob).
\end{lstlisting}
\end{minipage}

does not return a generalized hypothesis but the examples
\begin{lstlisting}
woman(ann).
woman(mary).
\end{lstlisting}.
After including the determination declaration
\begin{lstlisting}
:- determination(woman/1,female/1).
\end{lstlisting}
a more general hypothesis \tc{woman(A) :- female(A).} is learnt. To the author it is not clear why the hypothesis space has to be defined with two definitions. Systems like Imparo do not need determinations declarations.

\subsection{Assumption of consistency}\label{aleph_consistency_assumption}
Aleph assumes the examples are consistent.

\begin{minipage}[t]{.30\textwidth}
\begin{lstlisting}
%background theory
\end{lstlisting}
\end{minipage}
\begin{minipage}[t]{.30\textwidth}
\begin{lstlisting}
%positive examples
woman(ann).
\end{lstlisting}
\end{minipage}
\begin{minipage}[t]{.30\textwidth}
\begin{lstlisting}
%negative examples
woman(ann).
\end{lstlisting}
\end{minipage}

returns a hypothesis \tc{woman(ann).} which is inconsistent with the negative examples. Aleph assumes that the set of background knowledge union with examples is a consistent.

\subsection{Unlearnability of regular languages}\label{aleph_unlearnability_of_regular_languages}
Although Aleph can learn a simple term structure, it cannot learn more complex concepts like a regular language represented by regular expression \tc{(ss)*}.

\begin{lstlisting}
:-modeh(*, in_language(s(s(+word)))).
:-modeh(*, in_language(s(+word))).
:-modeb(*, in_language(+word)).
:- determination(in_language/1, in_language/1).
\end{lstlisting}

\begin{minipage}[t]{.50\textwidth}
\begin{lstlisting}
%background theory
in_language(epsilon).
in_language(s(s(epsilon))).
in_language(s(s(s(s(epsilon))))).
\end{lstlisting}
\end{minipage}
\begin{minipage}[t]{.20\textwidth}
\begin{lstlisting}
%positive examples
in_language(s(s(epsilon))).
in_language(s(s(s(s(epsilon))))).

%negative examples
in_language(s(epsilon)).
in_language(s(s(s(epsilon)))).
\end{lstlisting}
\end{minipage}

returns back positive examples
\begin{lstlisting}
in_language(s(s(epsilon))).
in_language(s(s(s(s(epsilon))))).
\end{lstlisting}
A more general hypothesis representing the language is
\begin{lstlisting}
in_language(s(s(A)) :- in_language(A).
\end{lstlisting}
If our target hypothesis had only one function symbol in its predicate and its correspondent examples, then a simpler hypothesis
\begin{lstlisting}
in_language(s(A)) :- in_language(A).
\end{lstlisting}
could indeed be learnt.

\subsection{Unlearnability of multi-clausal concepts}
Aleph cannot learn a hypothesis that needs more that one clause to explain the examples.

\begin{minipage}[t]{.50\textwidth}
\begin{lstlisting}
:-modeh(*, woman(+person)).
:-modeh(*, man(+person)).

:-modeb(*, male(+person)).
:-modeb(*, female(+person)).

:- determination(woman/1, female/1).
:- determination(man/1, male/1).

%positive examples
man(jack).
man(martin).
woman(jane).
woman(susan).
\end{lstlisting}
\end{minipage}
\begin{minipage}[t]{.20\textwidth}
\begin{lstlisting}
%background theory
female(jane).
female(susan).
female(eugenia).

couple(jack, jane).
couple(sam, susan).
couple(martin, eugenia).

male(X) :- couple(X,Y), woman(Y).

%negative examples
man(jane).
woman(sam).
\end{lstlisting}
\end{minipage}

produces a hypothesis
\begin{lstlisting}
man(jack).
man(martin).
woman(A) :- female(A).
\end{lstlisting}

However, if we put the induced hypothesis to the background knowledge, then a hypothesis generalizing \tc{man} examples would be produced
\begin{lstlisting}
man(A) :- male(A).
woman(A) :- female(A).
\end{lstlisting}

To explain the \tc{man} examples we needed to learn a two clausal hypothesis. This limitation is present in Toplog, however not in Imparo. The limitation not only demonstrates inability to learn more complex concepts, but also to make deductions from the observations such as \tc{woman} examples in order to deduce a hypothesis.

\subsection{Other assumed limitations}
Any hypothesis learnt in Aleph has to be a Horn theory. Aleph cannot learn second-order logic statements. Aleph cannot make deductions from the observations as it can from the background knowledge.

\section{Aleph's violations and biases}

\subsection{Default Example Bias}\label{aleph_default_example_bias}
Aleph includes examples in its bias even if these are not specified by mode declarations.

\begin{minipage}[t]{.50\textwidth}
\begin{lstlisting}
%:-modeh(*, woman(+person)).
%:-modeb(*, female(+person)).
\end{lstlisting}
\end{minipage}
\begin{minipage}[t]{.20\textwidth}
\begin{lstlisting}
%positive examples
woman(ann).
\end{lstlisting}
\end{minipage}

returns a hypothesis
\begin{lstlisting}
woman(ann).
\end{lstlisting}

Therefore the hypotheses space is not entirely defined by mode declarations and determination declarations.

\subsection{Weak head mode declaration}\label{aleph_weak_head_mode_declarations}
A head mode declaration in Aleph allows only definitions with specific term structure as opposed to
the ability to substitute a variable or a type (e.g. \tc{+person}) for any term.

\begin{minipage}[t]{.55\textwidth}
\begin{lstlisting}
:-modeh(*, woman(+person)).
:-modeb(*, anybody(+person)).
:- determination(woman/1, anybody/1).
person(sister(A)).

%background theory
anybody(jane).
anybody(jack).
\end{lstlisting}
\end{minipage}
\begin{minipage}[t]{.30\textwidth}
\begin{lstlisting}
%positive examples
woman(sister(jane)).
woman(sister(jack)).

%negative examples
woman(sister(frog)).
\end{lstlisting}
\end{minipage}

returns back the positive examples
\begin{lstlisting}
woman(sister(jane)).
woman(sister(jack)).
\end{lstlisting}

However, if we added the \tc{sister} function symbol in the mode declaration explicitly
\begin{lstlisting}
:-modeh(*, woman(sister(+person))).
\end{lstlisting}
then a more general hypothesis
\begin{lstlisting}
woman(sister(A)) :- anybody(A).
\end{lstlisting}
would be learnt. This constraint is present in Imparo as well. To the author the reason is not clear since \tc{woman(sister(+person))} is subsumed by
\tc{woman(+person)} which was already present in the mode declaration. Therefore unless specified in the mode declarations, Aleph can learn only the hypothesis whose terms do not contain the function symbols.

\subsection{Literal observation bias}
Aleph can accept only the examples in the form of literals. If presented with other clauses in the example file, it will not run

\begin{minipage}[t]{.50\textwidth}
\begin{lstlisting}
:-modeh(*, man(+person)).
:-modeb(*, male(+person)).
:- determination(man/1, male/1).

%background theory
male(jack).
male(sam).
male(john).
male(tristan).
female(glory).
\end{lstlisting}
\end{minipage}
\begin{minipage}[t]{.20\textwidth}
\begin{lstlisting}
%positive examples
man(jack) :- male(jack).
man(sam) :- male(sam).
man(john) :- male(john).

%negative examples
man(glory).
\end{lstlisting}
\end{minipage}

Expected hypothesis explaining the examples:
\begin{lstlisting}
man(A) :- male(A).
\end{lstlisting}
In comparison systems like Imparo and Toplog do not accept examples of non-literal form either.

\subsection{Preference over earlier determinations}\label{aleph_preference_over_earlier_determinations}
Aleph prefers learning a hypothesis whose determination declaration has been defined earlier.

\begin{minipage}[t]{.60\textwidth}
\begin{lstlisting}
:-modeh(*, man(+person)).
:-modeb(*, male(+person)).
:-modeb(*, bridegroom(+person)).

:- determination(man/1, bridegroom/1).
:- determination(man/1, male/1).

bridegroom(jack).
bridegroom(sam).
bridegroom(john).
\end{lstlisting}
\end{minipage}
\begin{minipage}[t]{.20\textwidth}
\begin{lstlisting}
%background theory
male(jack).
male(sam).
male(john).

%positive examples
man(jack).
man(sam).
man(john).

%negative examples
man(susan).
\end{lstlisting}
\end{minipage}

returns a hypothesis
\begin{lstlisting}
man(A) :- bridegroom(A).
\end{lstlisting}

Changing the order of the determination declarations to
\begin{lstlisting}
:- determination(man/1, male/1).
:- determination(man/1, bridegroom/1).
\end{lstlisting}
results in a hypothesis
\begin{lstlisting}
man(A) :- male(A).
\end{lstlisting}
In both cases, the induced hypothesis had a head atom declared in an earlier determination. This is an opposite search bias to the Imparo's preference over the later mode declarations.
