\chapter{Appendix - evaluation of Xhail on examples}\label{appendix_xhail}
\section{Xhail's capabilities}

\subsection{Learnability of regular languages}\label{xhail_learnability_regular_languages}
Xhail can learn regular languages of the form \tc{(ss)*} where \tc{*} represents a Kleene star operator.

\begin{minipage}[t]{.70\textwidth}
\begin{lstlisting}
modeh(0,10,max,in_language(s(s("+word")))).
modeb(0,10,pos,in_language("+word")).
goal(g).
in_language(e).
g:- in_language(s(s(e))), not in_language(s(e)).
\end{lstlisting}
\end{minipage}
\begin{minipage}[t]{.20\textwidth}
\begin{lstlisting}
word(e).
word(s(e)).
word(s(s(e))).
word(s(s(s(e)))).
word(s(s(s(s(e))))).
\end{lstlisting}
\end{minipage}

produces the expected hypothesis \tc{in\_language(s(s(X1)):-in\_language(X1)} where \tc{e} is an empty word in a language.

\subsection{Generalization downwards}\label{xhail_generalization_downwards}
Xhail can learn a hypothesis of the form $P(x) \leftObjectImplies P(s(x))$.

\begin{minipage}[t]{.60\textwidth}
\begin{lstlisting}
modeh(0,10,max,negative("+int")).
modeb(0,10,pos,negative(s("+int"))).
goal(g).
negative(s(-2)).
g:- negative(-2), not negative(s(s(-2))).
\end{lstlisting}
\end{minipage}
\begin{minipage}[t]{.20\textwidth}
\begin{lstlisting}
int(-2).
int(s(-2)).
int(s(s(-2))).
int(s(s(s(-2)))).
int(s(s(s(s(-2))))).
\end{lstlisting}
\end{minipage}

returns the expected hypothesis \tc{negative(X1):-negative(s(X1)),int(X1)}.

\subsection{Goal examples}
Xhail specifies examples in a goal to be learnt.

If one wanted to translate learning from examples problem into the Xhail's learning problem, positive and negative examples could be placed in a goal:

\begin{minipage}[t]{.50\textwidth}
\begin{lstlisting}
modeh(0,10,max,woman("+person")).
modeb(0,10,pos,female("+person")).

person(susan; adam).
\end{lstlisting}
\end{minipage}
\begin{minipage}[t]{.20\textwidth}
\begin{lstlisting}
female(susan).
male(adam).
goal(g).
g:- woman(susan), not woman(adam).
\end{lstlisting}
\end{minipage}

where \tc{woman(susan)} is a positive example and a \tc{woman(adam)} is a negative example. The hypothesis learnt is
\tc{woman(X1) :- female(X1), person(X1).}.

\subsection{Multiple hypotheses}
Like Tal, Xhail returns multiple possible hypotheses (or solutions) for a given learning problem.

\begin{minipage}[t]{.50\textwidth}
\begin{lstlisting}
modeh(0,1,pos,e).
modeb(0,1,pos,a).
modeb(0,1,pos,b).
find(all).
\end{lstlisting}
\end{minipage}
\begin{minipage}[t]{.20\textwidth}
\begin{lstlisting}
%background
a.
b.
e.
\end{lstlisting}
\end{minipage}

returns the hypotheses

\begin{minipage}[t]{.25\textwidth}
\begin{lstlisting}
1. e :- true.
2. e :- b.
\end{lstlisting}
\end{minipage}
\begin{minipage}[t]{.20\textwidth}
\begin{lstlisting}
3. e :- a.
4. e :- a,b.
\end{lstlisting}
\end{minipage}

\subsection{Multi-clausal hypotheses}
Xhail can learn a hypothesis that consists of multiple clauses:

\begin{minipage}[t]{.50\textwidth}
\begin{lstlisting}
modeh(0,10,all,woman("+person")).
modeh(0,10,all,man("+person")).
modeb(0,10,pos,female("+person")).
modeb(0,10,pos,male("+person")).

g:- woman(susan), not woman(adam), man(adam), not man(susan).
\end{lstlisting}
\end{minipage}
\begin{minipage}[t]{.20\textwidth}
\begin{lstlisting}
goal(g).
person(susan; adam).
female(susan).
male(adam).
\end{lstlisting}
\end{minipage}

returns a hypothesis
\begin{lstlisting}
man(X1):-male(X1), person(X1).
woman(X1):-female(X1), person(X1).
\end{lstlisting}.
One clause would be insufficient to explain both \tc{man} and \tc{woman} examples.

\subsection{Fine search space control}\label{xhail_fine_search_space_control}
Xhail has extended mode declarations.

\begin{minipage}[t]{.50\textwidth}
\begin{lstlisting}
modeh(0,1,all,woman("+person")).
modeb(0,1,pos,female("+person")).
person(susan; ruth; adam).

goal(g).
g:- woman(susan), woman(ruth), not woman(adam).
\end{lstlisting}
\end{minipage}
\begin{minipage}[t]{.20\textwidth}
\begin{lstlisting}
%Background
female(susan).
female(ruth).male(adam).
\end{lstlisting}
\end{minipage}

returns no hypothesis. However, placing only one positive example in a goal
\tc{g:- woman(susan), not woman(adam).} or modifying the modeh declaration to cover the two positive examples\\
\tc{modeh(0,2,all,woman("+person")).} makes Xhail to learn a hypothesis\\
\tc{woman(X1):-female(X1),person(X1).}.

Apart from the extended mode declarations, additional control in Xhail has been introduced with \tc{goal}, \tc{find}, \tc{task} statements:

\begin{quote}\cite{ray2007xhail}
\emph{Zero or more goal statement} of the form \tc{goal(a).} means to only compute models satisfying the ground atom \tc{a}.

\emph{Zero or one find statement} is of the form \tc{find(all)="all solutions"} or \tc{find(min)="minimal solutions"}.

\emph{Zero or one task declarations} are possible, e.g. \tc{task(abduce).} or \tc{task(induce).}.
\end{quote}

\subsection{Learning by integrity constrains}
Xhail can learn from the background knowledge in a form of integrity constrains.

\begin{minipage}[t]{.30\textwidth}
\begin{lstlisting}
modeh(0,1,all,a).
modeh(0,1,all,r).
\end{lstlisting}
\end{minipage}
\begin{minipage}[t]{.20\textwidth}
\begin{lstlisting}
goal(p).
\end{lstlisting}
\end{minipage}
\begin{minipage}[t]{.20\textwidth}
\begin{lstlisting}
p :- r.
false :- p, not a.
\end{lstlisting}
\end{minipage}

returns a hypothesis \tc{a:-true. r:-true.} where in the learning problem
\tc{false :- p, not a.} represents an integrity constraint.

\subsection{Term structure learnability}
Xhail can learn the hypotheses whose terms contain function symbols.

\begin{minipage}[t]{.60\textwidth}
\begin{lstlisting}
modeh(0,10,all,woman(sister("+a"))).
modeb(0,10,pos,anybody("+a")).

a(susan; adam; frog).
anybody(susan; adam).
goal(g).
g:- woman(sister(susan)), woman(sister(adam)), not woman(sister(frog)).
\end{lstlisting}
\end{minipage}
\begin{minipage}[t]{.20\textwidth}
\begin{lstlisting}
female(susan).
male(adam).
\end{lstlisting}
\end{minipage}

returns a hypothesis
\tc{woman(sister(X1)):-anybody(X1), a(X1).}.
\subsection{Other capabilities}
Xhail can learn a hypothesis which requires ground terms to be in a head predicate by specifying such a hypothesis with the constant type \tc{\#type} in mode declarations. Xhail supports hypotheses containing predicates of arities greater than 1.

\section{Xhail's assumed limitations}

\subsection{Partial explanations impossible}
Xhail cannot learn a hypothesis that explains some positive examples, but not all.

\begin{minipage}[t]{.60\textwidth}
\begin{lstlisting}
modeh(0,10,all,woman("+person")).
modeb(0,10,pos,female("+person")).
person(susan; ruth; adam).
goal(g).
g:- woman(susan), woman(ruth), not woman(adam).
\end{lstlisting}
\end{minipage}
\begin{minipage}[t]{.20\textwidth}
\begin{lstlisting}
female(susan).
male(adam).
\end{lstlisting}
\end{minipage}

returns no hypothesis.Although a hypothesis
\tc{woman(X1):-female(X1),person(X1).} cannot explain the example \tc{woman(ruth).} it would be able to explain the example \tc{woman(susan)}.

In comparison, Aleph can learn partial hypotheses: from the background knowledge\\
\tc{female(alice).female(mary).}, positive examples \tc{woman(alice).woman(mary).woman(susan).}, negative examples \tc{woman(adam).} Aleph includes in its hypothesis a partial explanation
\tc{woman(A) :- female(A).}.

\subsection{Inability to learn from unrestricted clauses}\label{xhail_inability_to_learn_from_unrestricted_clauses}
Xhail requires that all its clauses are restricted by ground instances - a quantification over all possible ground instances in a domain is not possible.

An attempt to define that everybody in a domain is of type person fails. \tc{person(X1).} returns an error "a non-ground fact person(X1). All terms within a fact must be ground terms." For a clause \tc{person(X1) :- true.} an error message "Unrestricted variable X1." is produced. Therefore we always have to ground the clauses (containing variables) in Xhail, for example
\tc{grounded(susan; adam).person(X1) :- grounded(X1).} does not produce any complaints.
\section{Xhail's violations and biases}

\subsection{Implicit consistency check}\label{xhail_implicit_consistency_check}
Xhail does not explicitly report that the learning problem is inconsistent, it only outputs "NO SOLUTIONS" message which is also applicable for consistent problems whose model is not in the specified search space.

\subsection{Redundant hypotheses}\label{xhail_redundant_hypotheses}
Xhail returns multiple hypotheses even if they are the same.

\begin{minipage}[t]{.43\textwidth}
\begin{lstlisting}
modeh(1,3,min,fries("+food")).
modeb(1,3,neg,offer("+food")).
modeb(1,3,pos,offer("+food")).
modeb(1,3,neg,bistro("+food")).
modeb(1,3,pos,bistro("+food")).
\end{lstlisting}
\end{minipage}
\begin{minipage}[t]{.20\textwidth}
\begin{lstlisting}
goal(g).
food(md ; bk ; rz).
bistro(md ; bk ; rz).
offer(md ; bk).
meal(X):- food(X),fries(X), burger(X).
burger(X):- food(X), fries(X), offer(X).
g:- meal(md), meal(bk), not meal(rz).
\end{lstlisting}
\end{minipage}

returns the hypotheses

\begin{lstlisting}
1. fries(X1) :- food(X1).
2. fries(X1) :- food(X1).
\end{lstlisting}
where the 1st and the 2nd hypotheses are the same. Notice, a general hypothesis has been induced for every example, both \tc{md} and \tc{bk}.

\subsection{Other violations and biases}
Since Xhail learns multiple solutions, the hypotheses space is not biased towards the order of the mode declarations.

\begin{minipage}[t]{.50\textwidth}
\begin{lstlisting}
modeh(0,10,max,woman("+person")).
modeb(0,10,pos,female("+person")).
modeb(0,10,pos,cook("+person")).

g:- woman(susan), not woman(adam).
\end{lstlisting}
\end{minipage}
\begin{minipage}[t]{.20\textwidth}
\begin{lstlisting}
goal(g).
person(susan; adam).
female(susan).
cook(susan).
male(adam).
\end{lstlisting}
\end{minipage}

returns two hypotheses
\tc{1. woman(X1):-cook(X1), person(X1). 2. woman(X1):-female(X1), person(X1).}. In other systems a preference over the hypotheses is induced based on the order of the hypotheses' predicates \tc{cook(X1), female(X1)} in their modeb and determination declarations.
