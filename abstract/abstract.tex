
\addcontentsline{toc}{chapter}{Abstract}

\begin{abstract}
Inductive logic programming (ILP) is a subdiscipline of machine learning and logic programming. The goal of inductive logic programming is to study how the theory of inductive inference and logic programming can be used to design theoretical frameworks and programs solving machine learning problems. Inductive inference studies how to find a hypothesis explaining the observations from the known facts. 

An ILP system is a program that takes as an input a set of \emph{known} logical statements $B$ called background knowledge, a set of \emph{observed} logical statements $E$ called examples. The output of an ILP system is a set of logical statements $H$ called a \emph{hypothesis} explaining the observations from the background knowledge. Meaning $E$ can be deduced from $B$ and $H$, i.e. $B \cup H \models E$.

We classify 6 ILP systems: Progol, Aleph, Toplog, Xhail, Imparo and Tal. We assess systems experimentally on their ability to find a hypothesis on around 100 datasets of carefully constructed background knowledge and examples. Incompatibilities between ILP systems and their theoretical frameworks are reported resulting in incompleteness of ILP systems for finding a hypothesis with their complete theoretical frameworks.

As the search space of possible hypotheses for given background knowledge and examples can be large, ILP systems impose heuristics and systematic restrictions such as mode declarations and determinations to search its subset called a bias.
We formalise these notions encountered in implementations and for the first time establish a correspondence with a well-known theoretical formalism of a bias called a production field \cite{inoue1992linear}.

Yamamoto et al.\cite{yamamoto2012inverse} devise a complete method for finding a hypothesis with more efficient inverse subsumption as opposed to the standard method of anti-entailment. We extend the method to include negative examples and relax it while preserving its completeness. We encompass the theoretical results in an implementation of a new ILP system Rationale. Finally, we solve an open problem that \emph{Imparo is complete by inverse subsumption} proving that every correct hypothesis subsumes some connected theory (a set of logical statements constructed in Imparo's theoretical framework).
\end{abstract}
