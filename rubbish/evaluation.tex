
\section{Achievements wrt to initial goal}
The author apraises the work achieved by the chapters

The author outlines the turning points of the project timeline which relate the initial goals to what has been achieved and provide a more accurate assement of the final achievements taking into consideration the difficulties faced:

\begin{itemize}
\item classification of ILP systems as mathematical objects,
\item foundations of ILP,
\item retreat to the experimental classification of ILP systems,
\item work on the theory of ILP,
\item extended classification of ILP systems.
\end{itemize}

The author analyses the turning points and evaluates the contributions towards the initial goal.

\subsection{Classification of ILP systems as mathematical objects}
\subsection{Foundations of ILP}
\subsection{Retreat to the experimental classification of ILP systems}
\subsection{Work on the theory of ILP}
\subsection{Extended classification of ILP systems}

The project timeline is presented which is then subsequently used to reflect on the actual achievements relative to initial goals.

\section{Evaluation relative to initial goals}
Being interested in artificial intelligence, the initial goal of the author was to conduct a classification of ILP systems as mathematical structures from which an insight could be born into how to design an ILP system that would be able to solve learning problems at a level of human intelligence and greater.

The initial goal has remained unachieved. The reason was an inappropriate understanding of the ILP field by the author. The author constructed his plan of highly theoretical work in the classification of ILP systems based on the assumptions of the existence of the rigorous theoretical foundations in ILP. But the goal of ILP does not concern with searching a method for finding a solution to a general problem in the field of general artificial intelligence. This contrasts with the approach taken by Hutter by introducing a provably most intelligent (in a defined sense) agent in his work on Universal Artificial Intelligence \cite{hutter2005universal}
that uses the optimality of the Solomonoff's inductive inference 
\cite{solomonoff1964formal}\cite{legg1997solomonoff} which initially inspired the author into the classification of ILP systems as systems using inductive inference to learn from the environment.

Instead, the author has discovered that the goal of ILP (similarly to the goal of machine learning) is to design and build an ILP system that can return an expected solution for a set of well-understood problems (cf. \ref{evaluation_experimental_classification}) and then use such a system in a specific domain (such as molecular biology \cite{srinivasan1994mutagenesis}\cite{srinivasan1997carcinogenesis}) to bring a mainly practical value.

Consequently, the author retreated from his initial goals to an experimental classification of ILP systems and an analysis of ILP systems at the level of the existing work \cite{nienhuys1997foundations}\cite{muggleton1995inverse}\cite{yamamoto2012inverse} in the field.
