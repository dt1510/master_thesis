\subsection{ILP system definition}
An ILP system is a function
$f:\langle O, B, \mathcal{O}, \mathcal{B}, \mathcal{H}\rangle \mapsto H$ where the pentuple $\langle O, B, \mathcal{O}, \mathcal{B}, \mathcal{H}\rangle$
is an inductive logic programming problem and a hypothesis $H \in \mathcal{H}$ is a solution to an ILP problem.

We give an overview of ILP systems taking different approaches to an ILP problem. The definition of an ILP system we gave is not sufficient for making meaningful comparisons based on the usefulness of these systems in real applications nor from a theoretical viewpoint. The aim of familiarizing with these systems is to find the intuition on the key properties of ILP systems that should be formalized in order to benefit from the mathematical rigour required for reasoning about ILP systems.

Capability of predicate invention.
Based on T-directed framework. Unique T element. Unique bottom element.

\subsection{Aleph}

\subsection{Other systems}
Some other ILP systems include Aleph, Golem, Progol,
Spectre, EBG, Alecto, FOIL, Linus, Marvin, Mis, Confucius, Quinlan, ASPAL, Hyper, Tal, Tilde, Hail, CF-induction method.
