Formalisation of a language bias specification \ref{sec:background_language_bias}
Formalisation of determinations \ref{bias_determinations}
Formalisation of meta-constraints \ref{bias_meta-constraints}
Correspondence between mode declarations, determinations, meta constraints and production field: Proposition \ref{md_d_pf_correspondence_proposition}, Proposition \ref{proposition_meta-constraints_production_field}, Corollary \ref{bias_correspondence_corollary}
Classification by a language bias \ref{sec:classification_language_bias}: mode declarations, determinations and meta-constraints
Classification by a search bias \ref{classification_by_search_bias}

Experimental classification of ILP systems 
Unified definition of an ILP task \ref{unified_definition}
Classification by problem classes \ref{completeness_by_problem_classes} and robustness \ref{classification_robustness}
Appendix with around 100 experiments and comparisons of ILP systems with annotations

Theory developments in inverse subsumption
Extensions of inverse subsumption method with minimal complements:
to first-order theories \ref{Extension to first-order theories} and proof of its soundness: Proposition \ref{soundness_first_order_extension}
to negative examples \ref{extension_negative_examples}: algorithm \ref{negative_examples_antisubsumption_algorithm}, proof of its completeness and soundness: Proposition \ref{proposition_negative_examples_subsumption}
relaxation of an algorithm: \ref{relaxation_minimal_complements_algorithm} \ref{maximal_hypothesis_antisubsumer_algorithm}: proof of its completeness and soundness: Lemma \ref{lemma_hypothesis_wrt_bridge_theory_correspondence} Proposition \ref{completeness_of_inverse_subsumption_with_maximal_bridge_theory}
Completeness of inverse subsumption operators \ref{inverse_subsumption_operators_completeness}
Completeness of Imparo by inverse subsumption Theorem \ref{completeness_ctg}
