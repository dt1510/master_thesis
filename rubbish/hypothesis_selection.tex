\section{Hypothesis selection}
In \nameref{ilp_problem_definition} we required that an ILP problem has a unique solution. When formalising a definition that allowed several possible solutions we defined that such solutions were partial and the unique solution to the problem was a set of partial solutions.

For some learning objectives a set of solutions may be acceptable, for other learning objectives several possible solutions indicate that the problem has not been completely solved. This section analyses properties of a hypothesis, what it means for a hypothesis to have some specific property and how this information can be used to make a choice.

\subsection{Hypothesis properties}
We think of a property on a hypothesis as boolean valued function
$P:\mathcal{H} \to \{true, false\}$. We seek to list the intuitive notions of possible properties a hypothesis may have and formalise the most relevant ones.

\subsubsection{Hypothesis form}
An induction of hypotheses of certain forms tends to be statistically more successful than an induction of hypotheses of other forms.
Although one may hide the problem by introducing the Occam's razor to prefer the hypotheses of the simpler form, Goodman cautions in \cite{goodman1965new} that the form of the hypothesis depends on the description language as paraphrased:

\begin{cite}{goodman1965new}
Let H1 be a hypothesis that all emeralds are green. Let H2 be a hypothesis that all emeralds are grue - that is green until some time T in the future and blue afterwards. Then both H1 and H2 are (correct in ILP sense) explanations of our observations of green emeralds. However, we prefer inducing that all emeralds are green, not grue. The preference cannot be directly attributed to the complexity of the hypothesis. For suppose that the initial concepts are grue and brue - blue until the time T and green afterwards. Then H1 has a complex definition in terms of grue and brue whereas H2 is defined trivially by the initial concept grue.
\end{cite}

\subsection{Inducing preferences over hypotheses space}
\subsection{Hypothesis sufficiency}
\subsection{Hypothesis approximation}
