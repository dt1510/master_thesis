\section{Measure theory}
The following definition of a measure is adapted from Mathworld, Wolfram Web Resource\cite{wolframMathworldMeasure}.

\begin{defn}
A \emph{measure} is a real-valued function $\mu$ on a powerset $\powerset{S}$ of a set $S$ satisfying the following properties:
\begin{enumerate}
\item $\mu(\emptyset)=0$ and $\mu(S)=1$,
\item if $X \subseteq Y$ then $\mu(X) \le \mu(Y)$,
\item if $X_n, n=0, 1, 2, ...$ are pairwise disjoint, then
$\mu(\cup^\infty_{n=0} X_n)=\Sigma^\infty_{n=0}\mu(X_n)$
\end{enumerate}
\end{defn}

\begin{exmp}
Let $S$ be the set of possible throws of a die, $S=\{1, 2, 3, 4, 5, 6\}$. Define a measure $\mu:\powerset{S} \to \mathbb{R}$ over $\powerset{S}$ by $\mu(\{i\})=1/6, i \in S$. Then from the axioms of a measure it follows that $\mu(\emptyset)=0$, $\mu(S)=1$ and $\mu(\{1,2,5\})=1/2 \le \mu(\{1,2,3,4,5\})=5/6$. Such $\mu$ is called a probability measure and a triple $\langle \powerset{S}, S, \mu \rangle$ is called a probability space.
\end{exmp}

\begin{exmp}
Let $S$ be a real line $S=\mathbb{R}$, define function $\mu|_J$ on the set of intervals $J \subseteq \powerset{S}$ on the real line by $\mu|_J: (a, b) \mapsto b-a$, then $\mu|_J$ has a unique extension $\mu:S \to \mathbb{R}$ which is a measure function.
\end{exmp}

\begin{defn}
A logic programming language has a \emph{negation on failure}\cite{clark1978negation} NoF property iff $\forall \phi, \forall \psi. \phi \not\vdash \psi \implies \phi \vdash \neg\psi$ where the $\vdash$ is a consequence operator.
\end{defn}

\begin{exmp}
In a logic programming language Prolog in a program $P$ with a single sentence:

$spicy(X) :- curry(X).$

the atom $spicy(X)$ is not provable, therefore $P \vdash \neg spicy(X)$ since Prolog has a NoF property.
\end{exmp}

Notice that NoF property on the consequence operator directly implies its monotony. Unless indicated the logic programming language we will reason will will have a NoF property.

\begin{defn}
Let $\phi$ be a formula with free variables from
$\overline{v}=(v_{i_1},...,v_{i_m})$,
and let
$\overline{a}=(a_{i_1}, ..., a_{i_m})) \in M^m$.
We inductively define $\mathcal{M} \models \phi(\overline{a})$
as follows.\\
i) If $\phi$ is $t_1=t_2$, then $\mathcal{M} \models \phi(\overline{a})$ if $t_1^\mathcal{M}(\overline({a})=t_2^\mathcal{M}(\overline({a})$.\\
ii) If $\phi$ is $R(t_1, ..., t_{n_R})$, then $\mathcal{M} \models \phi(\overline{a})$ if $(t_1^\mathcal{M}(\overline{a}), ..., t_{n_R}^\mathcal{M}(\overline{a})) \in R^\mathcal{M}$.
iii) If $\phi$ is $\neg \psi$, then $\mathcal{M} \models \phi(\overline{a})$ if $\mathcal{M} \not\models \psi(\overline{a})$.
iv) If $\phi$ is $(\psi \land \theta)$, then $\mathcal{M} \models \phi(\overline{a})$ if $\mathcal{M} \models \psi(\overline{a})$ and 
\end{defn}
