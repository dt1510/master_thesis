The problem of finding $H$ for $B, E$ is called an \emph{ILP task}.

Central issues in ILP are:
\begin{itemize}
\item ILP task definition: what is the machine learning problem sought to be solved and how to define it,
\item completeness by problem classes: how to find a good explanation for every possible pair of $B, E$
\item hypothesis selection: what is a good explanation and how to choose one out of several possibilities,
\item hypothesis search: how to find a good hypothesis efficiently.
\end{itemize}

\section{Project development}
In order to classify ILP systems comprehensively, the author first seeked to learn the foundations of ILP from which one could build up a systematic way of comparison of ILP systems as mathematical objects. As the author to his great dissappointment realized that there were no foundations at the theoretical level required, the author decided to pursue his own development of the foundations. During the development, the author discovered that each ILP system solves a different ILP task (as a result of ILP systems not designed on the common foundations) and therefore ILP systems are incomparable on the majority of the properties characterising an ILP system as a mathematical object. Hence the author derived a definition of an ILP task unifying a small class of the properties of an ILP system and retreated to the experimental comparison of ILP systems. The author discoreved that ILP systems do not function as specified by their theoretical frameworks. Hence, in the end, the author pursued a development of the ideas on the theoretical frameworks based on the newest existing literature in hope one would be able to move the classification further at the theoretical level. During this stage the author extended the newest theoretical results and implemented an ILP system encompassing them. The author defined new concepts that were used previously intuitively or at an implementation level only which enabled him to extend the classification of ILP systems to the final state.
