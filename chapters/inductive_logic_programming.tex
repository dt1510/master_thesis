\chapter{Inductive logic programming}
The objective of this chapter is to present concepts from Inductive logic programming (ILP) and their extensions that are going to be used for the classification of ILP systems.

\section{Inductive logic programming}

\subsubsection{Inverse Entailment}
Inverse Entailment is a correspondence between an induction and a deduction:
\begin{thm}\cite{kimber2012learning}
Let $B$ be a Horn program, and let $h$ and $e$ be
Horn clauses. Then $B \wedge h \models e \iff B \wedge \neg e \models \neg h$.
\end{thm}

\paragraph{IE algorithm\cite{yamamoto2012inverse}}\label{inverse_entailement_algorithm}
Progol, Xhail, Imparo are based on the principle of the inverse entailment. By the principle of the inverse entailment $B \cup H \models E$ iff
$B \cup \neg E \models \neg H$. A hypothesis $H \in \mathcal{H}_2$ is a solution to the problem of the explanatory induction $(B,E^{+},E^{-},\mathcal{H})$ iff
$B \cup H \subseteq Cn(E^{+} \union \overline{E^{-}})$,
 $false \not\in Cn(E^{+} \cup E^{-})$, $\mathcal{H}_2 \subseteq \mathcal{H}$
iff
$B \cup H \subseteq Cn(E^{+} \union \overline{E^{-}})$,
 $false \not\in Cn(E^{+} \cup E^{-})$, $\mathcal{H}_2 \subseteq \mathcal{H}$
 
They compute the hypothesis $H$ in two steps:
1. constructing an intermediate theory, 2. generalizing its negation into the hypothesis with the inverse of the entailment relation.


\subsection{Input and output of ILP system}
An input to an ILP system is a logic program that can be conceptually divided into 3 parts:
\begin{itemize}
\item objectlevel statements: background knowledge, examples: positive and negative,
\item effective metalevel statements,
\item non-effective metalevel statements.
\end{itemize}

Objectlevel statements provide the main setting of a learning problem and concern with what is known: the language of enquiry $L$, a set of observations, etc.

Effective metalevel statements provide a refinement of the main setting and concern with how the learning problem may be solved and what form the solution is expected to take. Their specification converts a learning problem into a potentially easier, more directed and concrete \emph{ILP task}.
\emph{Mode declarations} and \emph{determinations} are effective metalevel statements specifying a constraint on a language $L$ to determine the search space $\mathcal{H}$ of possible hypotheses. The other effective metalevel statements may impose further \emph{metalevel constraints} on $L$. The set of constraints on $L$ determining the search space $\mathcal{H}$ is called a \emph{language bias}.
Apart from the specification of a language bias, effective metalevel statements contain search control statements that affect the form the computation of a hypothesis takes place and contain decision elements: preferential bias - which hypothesis should be preferred over the other, which parts of the background knowledge should be considered, etc.

Non-effective metalevel statements do not affect the result of the computation of a hypothesis. They are either required syntactic sugar or their primary purpose concerns different aspects of an ILP system, e.g. to provide the statistics about the computation to the user.

The primary output of an ILP system is a solution (a hypothesis or their set) to a refined learning problem.

\subsection{Language bias}\label{subsec:background_language_bias}
The search space of all possible hypotheses $H \in \mathcal{H}$ is restricted by a language bias $\mathcal{H}$. The main types of the biases in ILP are:
\begin{itemize}
\item mode declarations,
\item determinations,
\item metaconstrains,
\item induction and production field.
\end{itemize}

\subsubsection{Mode declarations}\label{background_mode_declarations}
\paragraph{Mode declaration in Progol}
Muggleton defines the hypothesis bias $\mathcal{H}$ with mode language and mode declarations.
\begin{defn}\cite{muggleton1995inverse}
A \emph{mode declaration} has either the form
\tc{modeh(n,atom)} or \tc{modeb(n,atom)} where $n$, the \emph{recall}, is either an integer, $n \ge 1$,
or \tc{*} and atom is a ground atom. Terms in the atom are either normal or placemarker. A normal term is either a constant or a function symbol followed by a
bracketed tuple of terms. A placemarker is either \tc{+type}, \tc{-type} or \tc{\#type}, where
type is a constant. If $m$ is a mode declaration then $a(m)$ denotes the atom of m
with place-markers replaced by distinct variables. The sign of $m$ is positive if $m$
is a \tc{modeh} and negative if $m$ is a \tc{modeb}.
\end{defn}

\begin{exmp}\cite{muggleton1995inverse}
\begin{lstlisting}
modeh(1,plus(+int,+int,-int))
modeb(*,append(-list,+list,+list)
modeb(1,append(+list,[+any],-list))
modeb(4,(+int ? #int))
\end{lstlisting}
\end{exmp}

\begin{remark}\cite{muggleton1995inverse}
The recall is used to bound the number of alternative solutions for instantiating
the atom. For simplicity, we assume in the following that all the modes have the
recall \tc{*}, meaning all solutions. The following defines when a clause is within Progol's definite mode language $L$.
\end{remark}

\begin{defn}\label{definition_definite_mode_language}\cite{muggleton1995inverse}
Let $C$ be a definite clause with a
defined total ordering over the literals and $M$ be a set of mode declarations.
$C = h \leftObjectImplies b1, ..., bn$ is in the \emph{definite mode language} $L(M)$ iff 1) $h$ is the atom
of a \tc{modeh} declaration in $M$ with every place-marker \tc{+type} and \tc{-type} replaced by
variables and every place-marker \tc{\#type} replaced by a ground term and 2) every
atom $b_i$ in the body of $C$ is the atom of a \tc{modeb} declaration in $M$ with every
place-marker \tc{+type} and \tc{-type} replaced by variables and every place-marker \tc{\#type}
replaced by a ground term and 3) every variable of \tc{+type} in any atom $b_i$ is either
of \tc{+type} in $h$ or of \tc{-type} in some atom $b_j$, $1 \le j < i$.
\end{defn}

\subsubsection{Determinations}
Some ILP systems like Aleph in addition to mode declarations use determination statements\cite{aleph2007} to impose a finer control on a search space.

\begin{defn}\cite{aleph2007}
A \emph{determination} is a logic statement of the form\\
\tc{determination(TargetName/TargetArity, BackgroundName/BackgroundArity).}
where \tc{TargetName}, \tc{BackgroundName} are predicate symbols and
\tc{TargetArity}, \tc{BackgroundArity} are their arities (non-negative integers) respectively.
\end{defn}

\begin{defn}\label{definition_determination_language}
Let $D=\langle d_1(tn_1/ta_1, bn_1/ba_1), ..., d_n(tn_n/ta_n, bn_n/ba_n)\ \rangle$ be an ordered list of determinations. A formula $H=h \leftObjectImplies b_1, ..., b_m$ is in \emph{determination language} $L(D)$ iff
1) $D \not=\emptyset$, 2) $tn_1=h$, 3) $\forall b_i (1 \ge i \ge m) \exists j (1 \ge j \ge n) bn_j=b_i$.
\end{defn}

\begin{exmp}
Let $D$ be the list of the determinations
\begin{lstlisting}
:− determination(man/ 1, male/1).
:− determination(man/ 1, bridegroom/1).
:− determination(woman/ 1, female/1).
\end{lstlisting}
Let the hypotheses be

\begin{minipage}[t]{.50\textwidth}
\begin{lstlisting}
H1=man(X) :- male(X).
H2=man(X) :- bridegroom(X).
H3=man(X) :- male(X), bridegroom(X).
\end{lstlisting}
\end{minipage}
\begin{minipage}[t]{.50\textwidth}
\begin{lstlisting}
H4=woman(X) :- female(X).
H5=man(X) :- male(X), female(X).
\end{lstlisting}
\end{minipage}

Then for the hypothesis bias (determination language) $\mathcal{H}=L(D)$,
$\{H_1, H_2, H_3, H_4, H_5\} \cap \mathcal{H}=\{H_1, H_2, H_3\}$.
\end{exmp}

\subsubsection{Metaconstraints}
Metaconstraints specify the additional constraints on the hypothesis space $\mathcal{H}$ not definable with the mode declarations and determinations.
An example being a metaconstraint $Cond$ specifying the maximum number of the literals allowed in a formula $H \in L$.

\begin{defn}
A \emph{metaconstraint} on a language $L$ is any formula $Cond$ defining some subset $L(Cond)$ (\emph{metaconstraint language}) of the language $L$, i.e. $L(Cond)=\{x \in L | Cond(x)\}\subseteq L$.
\end{defn}

Each ILP system has its own set of allowed metaconstrains.
\begin{exmp}
In Toplog \tc{set(maximum\_literals\_in\_hypothesis, 5)} defines clausal theories whose clauses consist of at most 5 literals.
In Imparo \tc{set\_max\_clauses(2)} defines clausal theories with at most 2 clauses.
In Tal \tc{option(max\_body\_literals, 5)} defines definite theories whose clauses have at most 5 positive literals.
\end{exmp}

\begin{proposition}
Let $Cond_1, Cond_2$ be metaconstrains on a language $L$.
Then the following are metaconstrains on the language $L$:
$\neg Cond_1$, $Cond_1 \land Cond_2$, $Cond_1 \lor Cond_2$.
\end{proposition}

\begin{proof}
Follows trivially from the definition of a metaconstraint.
\end{proof}

Therefore we will often think of a set of the metaconstraints to be a metaconstraint created from their conjunction.

\begin{defn}
Let $C=\{Cond_1, ..., Cond_n\}$ be a set of metaconstraints defining subsets of the language $L$.
A \emph{metaconstraint language} $L(C)$ is the intersection of the subsets defined by $Cond_1, ..., Cond_n$,
i.e. $L(C)=\{x \in L | Cond_1(x) \land ... \land Cond_n(x) \}$.
\end{defn}

\subsubsection{Induction and production field}
Inoue defines a production field as another form of a language bias.

\begin{defn}\cite{inoue2004induction}
A \emph{production field} $\mathcal{P}$ is a represented by a pair,
$\langle Lit, Cond\rangle$, where $Lit$
(\emph{the characteristic literals} of $\mathcal{P}$) is a subset of literals in a language $L$ and is closed under instantiation (that is, if a literal containing variables is in $Lit$, then all its instances are also in $Lit$), and $Cond$ is a certain condition to be satisfied. 
A clause $C$ is said to belong to a production field $\mathcal{P} = \langle Lit, Cond \rangle$ if every literal in $C$ belongs to $Lit$ and $C$ satisfies $Cond$.
\end{defn}

A special case of the production field when a condition $Cond \equiv true$ is called an induction field.

\begin{defn}\cite{yamamoto2012inverse}\label{induction_field_definition}
An \emph{induction field} is denoted by $\mathcal{I}_H = \langle Lit \rangle$,
where $Lit$ is a finite
set of literals to appear in ground hypotheses.
A ground hypothesis $H_g$ \emph{belongs to} $\mathcal{I}_H$ if
every literal in $H_g$ is included in $Lit$.
\end{defn}

\subsubsection{Correspondence between mode declarations, determinations, meta constraints and production field}
A production field is a generalization of mode declarations, determinations and metaconstraints.

\begin{proposition}\label{md_d_pf_correspondence_proposition}
Let $M$ be a set of mode declarations, $D$ an ordered list of determinations. Then there exist production fields
$\mathcal{P}_M$,
$\mathcal{P}_D$,
$\mathcal{P}_{M,D}$
such that for any clause $H \in L$:
1) $H \in L(M) \iff H \in \mathcal{P}_M$,
2) $H \in L(D) \iff H \in \mathcal{P}_D$,
3) $H \in L(M) \cap L(D) \iff H \in \mathcal{P}_{M,D}$.
\end{proposition}
The following proof is non-constructive.
\begin{proof}
Let $L$ be a logic language. By the definitios of definite mode language\ref{definition_definite_mode_language} and determination language\ref{definition_determination_language}, $L(M)$ and $L(D)$ are definable.
Let $Cond_M$, $Cond_D$ be defining formulas for $L(M)$ and $L(D)$, i.e. $L(M)=\{x \in L | Cond_M(x)\}$, $L(D)=\{x \in L | Cond_D(x)\}$.
Let $Cond_{M,D}=Cond_M \land Cond_D$. Let $Lit$ be a set of all literals of $L$ and let
$Lit_M=L(M) \cap Lit$,
$Lit_D=L(D) \cap Lit$,
$Lit_{M,D}=Lit_M \cap Lit_D$.
Then for every clause $H \in L$ the production fields
$\mathcal{P}_M=\langle Lit_M, Cond_M \rangle$,
$\mathcal{P}_D=\langle Lit_D, Cond_D \rangle$,
$\mathcal{P}_{M,D}=\langle Lit_{M,D}, Cond_{M,D} \rangle$
satisfy the conditions 1, 2, 3 respectively trivially.
\end{proof}

Given production fields defining the bias of mode declarations and determinations, productions fields with the additional metaconstrains can be constructed.

\begin{proposition}\label{proposition_metaconstraints_production_field}
Let $\mathcal{P}$ be any production field, $C$ a set of metaconstraints.
Then there exists a production field $\mathcal{P}_C$ such that
$\mathcal{P}_C=\mathcal{P} \cap L(C)$, i.e.
$\forall H \in L. H \in \mathcal{P}_C \iff (H \in \mathcal{P}) \land (H \in L(C))$.
\end{proposition}
\begin{proof}
Let $\mathcal{P}=\langle Lit, Cond \rangle$,
$C=\{Cond_1, ..., Cond_n\}$.
Define $\mathcal{P}_C=\langle Lit, Cond \land Cond_1 \land ... \land Cond_n \rangle$. Take an arbitrary $H \in L$. Denote $Lit(H)$ to be the set of the literals of the clausal form of $H$.
Then $H \in \mathcal{P}_C$ iff $Lit(H) \subseteq Lit$ and
$(Cond \land Cond_1 \land ... \land Cond_n)(H)$ iff
$Lit(H) \subseteq Lit$ and $Cond(H)$ and $Cond_1(H) \land ... \land Cond_n(H)$
iff $H \in \langle Lit, Cond \rangle = \mathcal{P}$ and $H \in L(Cond_1 \land ... \land Cond_n) = L(C)$. Thus $\mathcal{P}_C=\mathcal{P} \cap L(C)$.
\end{proof}

\begin{corollary}
Let $M$ be a set of mode declarations, $D$ an ordered list of determinations, $C$ a set of the metaconstraints.
Then there exist production fields
$\mathcal{P}_{M,C}$,
$\mathcal{P}_{M,D,C}$
such that for any clause $H \in L$:
1) $H \in L(M) \cap L(C) \iff H \in \mathcal{P}_{M,C}$,
2) $H \in L(M) \cap L(D) \cap L(C) \iff H \in \mathcal{P}_{M,D,C}$.
\end{corollary}
\begin{proof}
Follows from \ref{md_d_pf_correspondence_proposition} and \ref{proposition_metaconstraints_production_field}.
\end{proof}

The converse depends on the possible metaconstraints. If one allowed a metaconstraint for every possible definable condition $Cond$, then one could express any production field $\mathcal{P}=\langle Lit, Cond \rangle$ with mode declarations, determinations and metaconstraints.

\section{ILP systems}
An ILP system is a program that takes as an input background knowledge $B$ and examples $E$ and produces as an output an explanation called a hypothesis $H$ explaining the examples in terms of the background knowledge $B$, i.e. $E \subseteq Cn(B \cup H)$.
We give a list of ILP systems Progol, Aleph, Toplog, Xhail, Imparo, Tal.

\subsection{Progol}
We paraphrase the introduction to Progol from its official website\cite{muggleton1999progolWebsite}:
\begin{quote}
Progol combines Inverse Entailment with general-to-specific search through a refinement graph. Inverse Entailment is used with mode declarations to derive the most-specific clause within the mode language which entails a given example. This clause is used to guide a refinement-graph search. Progol's search has a provable guarantee of returning a solution having the maximum "compression" in the search-space. To do so it performs an admissible A*-like search, guided by compression, over clauses which subsume the most specific clause. Progol deals with noisy data by using the compression measure to trade-off the description of errors against the hypothesis description length. Progol allows arbitrary Prolog programs as background knowledge and arbitrary definite clauses as examples.
\end{quote}

\subsection{Aleph\cite{aleph2007}}
Aleph is an ILP system based on Progol developed in order to understand the concepts behind the inverse entailment\cite{muggleton1995inverse} influenced by the ideas from other ILP systems: CProgol, FOIL, FORS, Indlog, MIDOS, SRT, Tilde, and WARMR. 

\subsection{Toplog\cite{santos2008toplogWebsite}\cite{muggleton2008toplog}}
Toplog is an top-down ILP system implementing TDHD framework:
A set of clauses $\top$ called a top theory is required on the input in addition to observations and the background knowledge, subsequently the search space is restricted by requiring that each hypothetised clause of a hypothesis $H$ must be entailed by the top theory $\top$.

The Toplog algorithm used to construct the hypothesis uses Mode Directed Inverse Entailment and follows the steps:
\begin{itemize}
\item construct the top theory $\top$,
\item hypothesis derivation: derive refutations of $\neg e$ from $B$ and $\top$, derive a clause $h$ from the refutations, add $h$ to $H$.
\item coverage computation: which examples $E^+$ and $E^-$ are entailed by $h \in H$.
\item hypothesis construction: select $H' \subseteq H$ maximizing the score function - e.g. compression, coverage, accuracy,
\end{itemize}

\subsubsection{MC-Toplog\cite{muggleton2012mc}}
MC-Toplog is a sequel of Toplog, derives hypotheses like Toplog, in addition allows multiple clauses in a hypothesis. Its extended framework TDTcD restricts a hypotheses space to clauses entailing generalization of multiple examples (co-generalization) as opposesed to Toplog that could generalizing only a single example.

\subsection{Hail\cite{ray2003hybrid}\cite{ray2005phdHybrid}}
A Hail (Hybrid abductive inductive learning) ILP system extending the Progol's incomplete method of bottom generalization to a more complete method of \emph{kernel set generalization} with the methods of abductive logic programming.

\subsection{Xhail\cite{ray2009nonmonotonic}}
Xhail standing for eXtended Hybrid abductive learning is an ILP system extending Hail's methodology from Horn theories to normal logic programs.
Xhail version 2 was used for the experiments presented in this thesis.

\subsection{Imparo}\cite{kimber2012learning}
Imparo is an ILP system based on a general Induction on Failure theoretical framework extending the Hail's incomplete method of kernel set generalization to the complete method of connected theory generalization.
\subsubsection{Induction on Failure framework\cite{kimber2012learning}}
Induction on Failure framework (IoF) is a method for deriving a hypothesis $H$ where a single clause $h \in H$ does not necessarily need to explain an example $e \in E$, but an example can be explained by multiple clauses. Such a search space is called a connected theory.
\begin{defn}
A connected theory $T$ for a ground Horn clause $e$ and a Horn theory $B$ is a set of clauses that can be partitioned into sets $T_1, ..., T_n$ so that
(i) $B \union T_1^+ \models e_{head}$,
(ii) $\forall i \in \{1, ..., n-1\}. B \union e_{body} \union T_{i+1}^+ \models T_i^-$,
(iii) $B \union e_{body} \models T_n^-$,
(iv) $B \union T \not\models \square$.
\end{defn}

\subsubsection{Imparo\cite{kimber2012learning}}
Imparo uses the following algorithm to compute the hypothesis $H$:
\begin{itemize}
\item 1: select an example $E$ from the set of positive examples $E_{pos}$,
\item compute the most specific connected theory for an example $E$ and the background knowledge $B$,
\item search the lattice of sets of clauses subsuming the connected theory and choose the hypothesis $H$ with the highest score according to the score function such that $H \models E$,
\item add $H$ to $B$,
\item remove all $E' \in E_{pos}$ implied by new $B$, $B \models E'$.
\item if $E_{pos} = \emptyset$ finish, otherwise go to 1.
\end{itemize}

\subsection{Tal\cite{corapi2010inductive}\cite{corapi2011tal}}
Tal (Top-directed Abductive Learning) is a non-monotonic top-down ILP system complete for background theories and hypotheses as normal logic programs. 
Tal relies on mapping an ILP problem into an equivalent ALP one. This enables the
use of established ALP proof procedures and the specification of richer language bias with integrity constraints. The mapping provides a principled search space for an ILP problem, over which an abductive search is used to compute inductive solutions.

\subsection{Other systems}
Other ILP systems include Metagol, Golem, Spectre, EBG, Alecto, FOIL, Linus, Marvin, Mis, Confucius, Quinlan, ASPAL, Hyper,  Tilde, CF-induction.

\subsection{Selecting ILP systems for classification}
Classification of all ILP systems would be out of the scope of this project and would not provide much more value than a classification of well chosen ILP systems.
For selection of ILP systems for classification the following criteria were taken into the consideration:

\begin{itemize}
\item a popularity of an ILP system,
\item a novelty and contribution of an ILP system,
\item an access: an availability of a general purpose implementation, documentation, research findings, easiness of communication (in person vs. over distance) with the original authors and respective experts.
\end{itemize}

In the end the author has decided to classify Progol, Aleph, Toplog, Xhail, Imparo, Tal; Aleph being the best documented working system and others having been developed in the author's department with a general purpose implementation. The author would like to note based on the private communication with the respective experts that MC-Toplog does not have a general purpose implementation, a user needs to write a specific top theory for each learning problem and adapt the current implementation to support it. Metagol\cite{muggleton2014meta}, being a very interesting and novel system, similarly, as of time of classification did not have a general purpose implementation.

\section{Inverse subsumption for complete explanatory induction\cite{yamamoto2012inverse}}
This section paraphrases work by Yamamoto et al. \cite{yamamoto2012inverse} unless stated otherwise. For more examples, explanations and results, due to space constrains the reader is encouraged to consult the original source shall the need arise. The main result is a complete algorithm \emph{inverse subsumption with minimal complements} for deriving a hypothesis wrt $B, E$ by the method of the inverse subsumption in lieu of the antientailment.

\subsection{Preliminaries}\label{subsec:preliminaries}
\begin{defn}
Let $C$, $D$ be clauses, then $C$ \emph{(theta-)subsumes} $D$ iff there is a substitution $\theta$ such that $C \theta \subseteq D$. We denote the relation by $C \subsumes D$.
\end{defn}

\begin{defn}\label{definition_theory_subsumption}
Let $S$ and $T$ be two clausal theories. Then, $S$ \emph{(theory-)subsumes} $T$, denoted by $S \subsumes T$, if for any clause $D \in T$, there is a clause $C \in S$ such
that $C \subsumes D$. We denote by the inverse relation $\subsumed$ of the (theory-) subsumption, called \emph{anti-subsumption}.
\end{defn}

\begin{defn}
Let $S$ be a ground clausal theory $\{C_1, C_2,... ,C_n\}$ where each clause $C_i$ $(1 \le i \le n)= l_{i,1} \lor l_{i,2} \lor... \lor l_{i,m_i}$.
The \emph{complement} of $S$ is defined as
\\$\bar{S} = \{\neg l_{1,k_1} \lor \neg l_{2,k_2} \lor... \lor \neg l_{n,k_n} |
1 \le k_1 \le m_1 , 1 \le k_2 \le m_2,..., 1 \le k_n \le m_n\}$.
In case that $S$ is empty, $\bar{S}$ is defined as the set $\{\bot\}$ where $\bot$ is the empty clause.
\end{defn}

\begin{remark}
$S$ is a CNF formula such that $\bar{S} \equiv \neg S$.
\end{remark}

\begin{defn}
\emph{$\tau(S)$} denotes the clausal theory obtained by removing all the tautologies from a logic theory $S$.
\end{defn}

\begin{defn}
\emph{$\mu(S)$} denotes the clausal theory obtained by removing from $S$ all clauses that are properly subsumed by clauses in a logic theory $S$.
The \emph{minimal complement} of $S$ is $M(S)=\mu(\bar{S})$.
\end{defn}

\begin{exmp}
Let $S=\{a \lor b, b \lor c, \neg c\}$. Then
$\bar{S}=\{\neg a \lor \neg b \lor c, \neg a \lor \neg c \lor c, \neg b \lor c, \neg b \lor \neg c \lor c\}$.
$M(S)=\{\neg a \lor \neg c \lor c, \neg b \lor c\}$,
$\tau(\bar{S})=\{\neg a \lor \neg b \lor c, \neg b \lor c\}$.
\end{exmp}

The inverse subsumption with minimal complements computes a hypothesis $H$ wrt to the bridge theory $F$ and the induction field $\mathcal{I}_H$.

\begin{defn}\label{definition_bridge_theory}
Let $B$ and $E$ be a background theory and examples, respectively.
Let $F$ be a ground clausal theory. Then $F$ is a \emph{bridge theory} wrt $B$ and $E$ if
$B \land \bar{E} \models F$ holds. If no confusion arises, a bridge theory wrt $B$ and $E$ will simply be called a bridge theory.
\end{defn}

The author finds the notion of the maximal bridge theory useful in later chapters.
\begin{defn}\label{maximal_bridge_theory_definition}
A \emph{maximal bridge theory} wrt $B$ and $E$ is a theory $F = B \cup \bar{E}$.
\end{defn}
\begin{remark}
Every bridge theory is entailed by a maximal bridge theory.
\end{remark}

We next define the target hypotheses using the notion of an induction field $\mathcal{I}_H$\ref{induction_field_definition}, together with a bridge theory $F$ as follows:

\begin{defn}\label{definition_hypothesis_wrt_induction_field_bridge_theory}
Let $H$ be a hypothesis. $H$ is a \emph{hypothesis wrt $\mathcal{I}_H$ and $F$} if there is a ground hypothesis $H_g$ such that $H_g$ consists of instances from $H$,
$F \models \neg H_g$ and $H_g$ belongs to $\mathcal{I}_H$.
\end{defn}

\begin{defn}\emph{Tautologies of an induction field.}
Given an induction field $\mathcal{I}_H = \langle Lit \rangle$, $Taut(\mathcal{I}_H)$ is defined
as the set of tautologies $\{\neg A \land A | A \in Lit, \neg A \in Lit\}$.
\end{defn}

\subsection{Inverse subsumption with minimal complements}
The generalization procedure based on inverse subsumption with minimal complements is as follows:

\begin{defn}\label{inverse_subsumption_with_minimal_complements_algorithm}
Let $B, E$ and $\mathcal{I}_H = \langle L \rangle$ be a background theory, examples and an induction
field, respectively. Let $F$ be a bridge theory wrt $B$ and $E$. A clausal theory $H$ is derived
by \emph{inverse subsumption with minimal complements} from $F$ wrt $\mathcal{I}_H$ if $H$ is constructed as follows.
\begin{itemize}
\item Step 1. Compute $Taut(\mathcal{I}_H)$;
\item Step 2. Compute $\tau(M(F \cup Taut(\mathcal{I}_H)))$;
\item Step 3. Construct a clausal theory $H$ satisfying the condition:
$H \subsumes \tau(M(F \cup Taut(\mathcal{I}_H)))$.
\end{itemize}
\end{defn}

Inverse subsumption with minimal complements ensures the completeness for finding
hypotheses wrt $\mathcal{I}_H$ and $F$.

\begin{lemma}\label{yamamoto2012inverseLemma2}\cite{yamamoto2012inverse}
Let $B$, $E$ and $I_H$ be a background theory, examples and an induction field,
respectively. Let $F$ be a bridge theory wrt $B$ and $E$. For every hypothesis $H$ wrt $I_H$ and $F$, $H$ satisfies the following condition:

$H \subsumes \tau(M(F \cup Taut(I_H))$.
\end{lemma}

\begin{thm}\emph{Completeness of inverse subsumption with minimal complements} Let $B$, $E$ and $\mathcal{I}_H$ be a background theory, examples and an induction field,
respectively. Let $F$ be a bridge theory wrt $B$ and $E$. For every hypothesis $H$ wrt $I_H$ and $F$,
$H$ is derived by inverse subsumption with minimal complements from $F$ wrt $\mathcal{I}_H$.
\end{thm}
\begin{proof}
Follows from \fullref{yamamoto2012inverseLemma2}.
\end{proof}