\chapter{Advancing solutions to issues in ILP}\label{chap:advancing_solutions_to_issues_in_ilp}
The objective of this chapter is to provide extensions and future directions for the existing approaches to the issues in ILP by the ILP systems classified. Some of the extensions are used in the implementation of an ILP system Rationale described in \autoref{chap:rationale_ilp_system}.

\section{Inverse Subsumption for Complete Explanatory Induction}\cite{yamamoto2012inverse}
Influenced by the results in \cite{yamamoto2012inverse}, we seek further extensions.

\subsection{Extension to first-order theories}
A generalization of the antisubsumption results to the first-order clausal theories enables a more efficient computation of the minimal complement, especially if the number of potential tautologies is large and a computation of anti-subsumption since there is less need to search for a subsumed hypothesis as the hypothesis is already not ground.

\begin{proposition}(Soundness of the first-order extension)
Let $B$, $E$ and $I_H$ be a first-order background theory, examples and an induction field, respectively. Let $F$ be a bridge theory wrt $B$ and $E$. Suppose that $H \subsumes \tau(M(F \cup Taut(I_H)))$, then $B \cup H \models E$.
\end{proposition}
\begin{proof}
By subsumption $H \models \tau(M(F \cup Taut(I_H))) \equiv \neg F$. By the principle of the inverse entailment $B \cup \neg F \models E$, hence
$B \cup H \models E$.
\end{proof}

However, it is not clear whether the completeness is preserved in the first order case.

$\tau'(S)$ denotes the clausal theory obtained by removing every clause from $S$ that has a ground instance that factors to a tautology. $I_H$ is an induction field of not necessarily ground literals.

\begin{conjecture}\label{subsumptionConjectureFirstOrder}
Let $B$, $E$ and $I_H$ be a background theory, examples and an induction field, respectively. Let $F$ be a bridge theory wrt $B$ and $E$. For every hypothesis $H$ wrt $I_H$ and $F$, $H$ satisfies the following condition:
$H \subsumes \tau'(M(F \cup Taut(I_H)))$.
\end{conjecture}

\begin{exmp}
Let $E=\{r(y)\}$,
$B=\{\neg p(x) \vee r(x)\}$,
$H=\{s(x), \neg s(x) \vee p(x)\}$,
$I_H=\{s(x), \neg s(x), p(x)\}$.
Then $F=B \cup \neg E=\{\neg p(x) \vee r(x), \neg r(y) \}$.
$Taut(I_H)=\{s(x) \vee \neg s(x)\}$.
$M(F \cup Taut(I_H))=$
$\{p(x) \vee r(y) \vee \neg s(z), p(x) \vee r(y) \vee s(z),$
$\neg r(x) \vee r(y) \vee \neg s(z),\neg r(x) \vee r(y) \vee s(z) \}$.

$\tau'(M(F \cup Taut(I_H))=\{p(x) \vee r(y) \vee \neg s(z), p(x) \vee r(y) \vee s(z)\}$ which is subsumed by the hypothesis $H$.
\end{exmp}

\begin{exmp}
All theories need not be grounded. Adapting the previous example, if the examples were
$E_2=\{r(a), r(b)\}$ with the same hypothesis
$H=\{s(x), \neg s(x) \vee p(x)\}$, then
$\tau'(M(F_2 \cup Taut(I_H))=\{p(x) \vee r(a) \vee \neg s(z), p(x) \vee r(a) \vee s(z), p(x) \vee r(b) \vee \neg s(z), p(x) \vee r(b) \vee s(z) \}$ which is subsumed by $H$. A careful reader notices that the adapted $\tau'(M(F_2 \cup Taut(I_H))$ is just a partial instantiation of the previous
$\tau'(M(F \cup Taut(I_H))$ with the same substitution
$theta=\{a / y, b / y\}$ as used for the examples:
$\tau'(M(F_2 \cup Taut(I_H)) \theta=\tau'(M(F_2 \cup Taut(I_H))$,
$E_2 \theta = E$.
\end{exmp}

\subsection{Relaxation of minimal complements by a hypothesis subsumer}
We relax the procedure of the minimal complements by using a notion of a hypothesis subsumer while preserving the completeness for finding correct hypotheses wrt to $B, E, \mathcal{I}_H$.

\begin{defn}
Let $B$, $E$, $\mathcal{I}_H$ be background knowledge, examples, an induction field. Then a \emph{hypothesis subsumer} wrt $B, E, \mathcal{I}_H$ is a theory
$T=\tau(M(F \cup Taut(\mathcal{I}_H))$ (recall $M(S)=\mu(\bar{S})$) for a maximal bridge theory $F=B \land \bar{E}$.
\end{defn}

By the completeness of inverse subsumption with minimal complements, for every hypothesis $H$ wrt $F$ and $\mathcal{I}_H$ it holds $H \subsumes \tau(M(F \cup Taut(\mathcal{I}_H))$. Since a maximal bridge theory $F$ is used, if a hypothesis $H$ wrt $\mathcal{I}_H$ is correct wrt $B, E$, then it can be derived by inverse subsumption with minimal complements with a hypothesis subsumer.

\begin{defn}
A hypothesis $H$ is derived by \emph{inverse subsumption with minimal complements with a hypothesis subsumer} wrt $B, E, \mathcal{I}_H$ iff
$H \subsumes \tau(M(B \cup \bar{E} \cup Taut(\mathcal{I}_H))$.
\end{defn}

\subsubsection{Comparison with the original algorithm}
Minimal complements with a hypothesis subsumer is a special case of the original minimal complements algorithm where the bridge theory $F$ is required to be maximal, i.e. $F=B \cup \bar{E}$. The requirement for a maximal bridge theory $B \cup \bar{E}$ results in an in efficiency if the background knowledge is large. The advantage of the restricted algorithm is its simplicity and the preservation of completeness for finding a correct hypothesis $H$ wrt $B, E$, therefore providing an opportunity to improve the efficiency in a different direction.

\subsubsection{Hypothesis subsumer and connected theory}

\subsection{Negative examples}