\chapter{Advancing solutions to issues in inductive logic programming}

\section{Inverse Subsumption for Complete Explanatory Induction}\cite{yamamoto2012inverse}
Influenced by the results in \cite{yamamoto2012inverse}, we seek further extensions.

\subsection{Extension to first-order theories}
A generalization of the antisubsumption results to the first-order clausal theories enables a more efficient computation of the minimal complement, especially if the number of potential tautologies is large and a computation of anti-subsumption since there is less need to search for a subsumed hypothesis as the hypothesis is already not ground.

\begin{proposition}(Soundness of the first-order extension)
Let $B$, $E$ and $I_H$ be a first-order background theory, examples and an induction field, respectively. Let $F$ be a bridge theory wrt $B$ and $E$. Suppose that $H \subsumes \tau(M(F \cup Taut(I_H)))$, then $B \cup H \models E$.
\end{proposition}
\begin{proof}
By subsumption $H \models \tau(M(F \cup Taut(I_H))) \equiv \neg F$. By the principle of the inverse entailment $B \cup \neg F \models E$, hence
$B \cup H \models E$.
\end{proof}

However, it is not clear whether the completeness is preserved in the first order case.

$\tau'(S)$ denotes the clausal theory obtained by removing every clause from $S$ that has a ground instance that factors to a tautology. $I_H$ is an induction field of not necessarily ground literals.

\begin{conjecture}\label{subsumptionConjectureFirstOrder}
Let $B$, $E$ and $I_H$ be a background theory, examples and an induction field, respectively. Let $F$ be a bridge theory wrt $B$ and $E$. For every hypothesis $H$ wrt $I_H$ and $F$, $H$ satisfies the following condition:
$H \subsumes \tau'(M(F \cup Taut(I_H)))$.
\end{conjecture}

\begin{exmp}
Let $E=\{r(y)\}$,
$B=\{\neg p(x) \vee r(x)\}$,
$H=\{s(x), \neg s(x) \vee p(x)\}$,
$I_H=\{s(x), \neg s(x), p(x)\}$.
Then $F=B \cup \neg E=\{\neg p(x) \vee r(x), \neg r(y) \}$.
$Taut(I_H)=\{s(x) \vee \neg s(x)\}$.
$M(F \cup Taut(I_H))=$
$\{p(x) \vee r(y) \vee \neg s(z), p(x) \vee r(y) \vee s(z),$
$\neg r(x) \vee r(y) \vee \neg s(z),\neg r(x) \vee r(y) \vee s(z) \}$.

$\tau'(M(F \cup Taut(I_H))=\{p(x) \vee r(y) \vee \neg s(z), p(x) \vee r(y) \vee s(z)\}$ which is subsumed by the hypothesis $H$.
\end{exmp}

\begin{exmp}
All theories need not be grounded. Adapting the previous example, if the examples were
$E_2=\{r(a), r(b)\}$ with the same hypothesis
$H=\{s(x), \neg s(x) \vee p(x)\}$, then
$\tau'(M(F_2 \cup Taut(I_H))=\{p(x) \vee r(a) \vee \neg s(z), p(x) \vee r(a) \vee s(z), p(x) \vee r(b) \vee \neg s(z), p(x) \vee r(b) \vee s(z) \}$ which is subsumed by $H$. A careful reader notices that the adapted $\tau'(M(F_2 \cup Taut(I_H))$ is just a partial instantiation of the previous
$\tau'(M(F \cup Taut(I_H))$ with the same substitution
$theta=\{a / y, b / y\}$ as used for the examples:
$\tau'(M(F_2 \cup Taut(I_H)) \theta=\tau'(M(F_2 \cup Taut(I_H))$,
$E_2 \theta = E$.
\end{exmp}

\subsection{Extension to cover negative examples}
We extend the completeness results and algorithm in a more general problem of explanatory induction that includes the negative examples.
\begin{defn}General ILP Problem setting: given the clausal theories $B$, $H$, $E^+$, $E^-$ find a hypothesis $H$ such that $B \wedge H \models E+$, $B \wedge H$ consistent, $B \wedge H \not\models E-$.
\end{defn}

\begin{lemma}\label{yamamoto2012inverseLemma2Converse}
Let $B \cup \neg E \models F$. $H \subsumes \tau(M(F \cup Taut(I_H)))$ implies
$B \cup H \models E$.
\end{lemma}
\begin{proof}
By the principle of the inverse entailment and by $\tau(M(F \cup Taut(I_H))) \equiv \neg F$.
\end{proof}

\begin{lemma}
Let $E-=\{e_1,...,e_n\}$, $F_i-=B \cup \{\neg e_i\}$ and $H$ be
a hypothesis wrt $I_H, B, E+, E-$.
Then $B \cup H \not\models E-$ iff
$\forall i \in \{1,...,n\}.$ $H \not\subsumes \tau(M(F_i- \cup Taut(I_H)))$.
\end{lemma}

\begin{proof}
Use \ref{yamamoto2012inverseLemma2}, \ref{yamamoto2012inverseLemma2Converse} and the fact that $E-$ is in a disjunctive normal form.
\end{proof}

\begin{conjecture}
$\tau(M(F_i- \cup Taut(I_H))) \subsumes \tau(M(F- \cup Taut(I_H)))$
where $F-=F_1- \wedge ... \wedge F_n-$.
\end{conjecture}

By the principle of inverse entailment
$B \wedge H \not\models E-$ iff
$B \wedge \neg E- \not\models \neg H$.

\begin{defn}
(Negative bridge theory) Let $B$ and $E-$ be a background theory and negative examples, respectively. Let $F-$ be a ground clausal theory. Then $F-$ is a negative bridge theory wrt $B$ and $E$ iff $B \wedge \neg E- \models F-$ holds.
\end{defn}

\begin{conjecture}\label{subsumptionConjectureNegativeExamples}
(Note: probably false).
Let $B$, $E-$ and $I_H$ be a background theory, negative examples and an induction field, respectively. Let $F-$ be a negative bridge theory wrt $B$ and $E-$. For every hypothesis $H$ wrt $I_H$ and $F-$,

$B \wedge H \not\models E-$ iff $H$ does not subsume
$\tau(M(F- \cup Taut(I_H)))$.
\end{conjecture}

\begin{defn}
Let $B$, $E+$, $E-$ and $I_H = \langle L \rangle$ be a background theory, positive examples, negative examples and an induction field, respectively.
Let $F$ be a bridge theory wrt $B$ and $E+$. Let $F-$ be a maximal negative bridge theory wrt $B$ and $E-$. A clausal theory $H$ is derived
by extended inverse subsumption with minimal complements from $F$ and $F-$ wrt $I_H$ if $H$ is constructed as
follows:
\begin{itemize}
\item Step 1. Compute $Taut(I_H)$;
\item Step 2. Compute $\tau(M(F \cup Taut(I_H)))$;
\item Step 3. Compute $\tau(M(F- \cup Taut(I_H)))$;
\item Step 4. Construct a clausal theory $H$ satisfying the conditions:
$H \subsumes \tau(M(F \cup Taut(I_H)))$,
$H \not\subsumes \tau(M(F- \cup Taut(I_H)))$.
\end{itemize}
\end{defn}