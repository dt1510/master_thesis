\chapter{Rationale ILP system}
An ILP system Rationale is designed and implemented based on the inverse subsumption with minimal complements algorithm \fullref{inverse_subsumption_with_minimal_complements_algorithm} by Yamamoto\cite{yamamoto2012inverse} and suggestions in \fullref{chap:advancing_solutions_to_issues_in_ilp}. Rationale solves an extension of an ILP task of explanatory induction for clausal theories $B, E^+, E^-, H$. The objective of this chapter is to reveal the details of the prototype implementation and explain the rationales behind.

\section{Algorithm overview}
Rationale uses a default maximal bridge theory $F=B \land \neg{E}$. By the completeness of inverse subsumption with minimal complements, for every hypothesis $H$ wrt $F$ and $\mathcal{I}_H$ it holds $H \subsumes \tau(M(F \cup Taut(\mathcal{I}_H))$. Therefore the steps of the Rationale algorithm are:
\begin{itemize}
\item Step 1. compute the bridge theory $F$,
\item Step 2. compute the $Taut(\mathcal{I}_H)$.
\item Step 3. ... etc.
\end{itemize}

\section{Bias}

\section{Completeness}

\section{Hypothesis selection}

\section{Hypothesis search}

\section{ILP task definition}

\section{Implementation details}
The yet uncovered details of the Rationale implementation are summarised.

\subsection{Language choice}

\subsection{Libraries used}

\section{Comparison of Rationale with other ILP systems}