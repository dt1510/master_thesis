\chapter{Rationale ILP system}
An ILP system Rationale is designed and implemented based on the work by Yamamoto\cite{yamamoto2012inverse} and suggestions in \fullref{chap:advancing_solutions_to_issues_in_ilp}. Rationale solves an extension of an ILP task of explanatory induction for clausal theories $B, E^+, E^-, H$. The objective of this chapter is to reveal the details of the prototype implementation and explain the rationales behind.

\section{Inverse subsumption with minimal complements algorithm}\cite{yamamoto2012inverse}
Rationale derives a hypothesis $H$ by inverse subsumption with minimal complements\cite{yamamoto2012inverse} which uses an induction field bias. This section presents the work by Yamamoto as cited, not the author's.

\begin{defn}\label{induction_field_definition}
An \emph{induction field}, denoted by $\mathcal{I}_H = \langle L \rangle$,
where $L$ is a finite
set of literals to appear in ground hypotheses.
A ground hypothesis $H_g$ \emph{belongs to} $\mathcal{I}_H$ if
every literal in $H_g$ is included in $L$.
Given an induction field $\mathcal{I}_H = \langle L \rangle$, $Taut(\mathcal{I}_H)$ is defined
as the set of tautologies $\{\neg A \land A | A \in L, \neg A \in L\}$.
\end{defn}

We next define the target hypotheses using the notion of an induction field $\mathcal{I}_H$ , together with a bridge theory $F$ as follows:

\begin{defn}
Let $H$ be a hypothesis. $H$ is a \emph{hypothesis wrt $\mathcal{I}_H$ and $F$} if there is a ground hypothesis $H_g$ such that $H_g$ consists of instances from $H$,
$F \models \neg H_g$ and $H_g$ belongs to $\mathcal{I}_H$.
\end{defn}

Now, the generalization procedure based on inverse subsumption with minimal comple
ments is as follows:

\begin{defn}
Let $B, E$ and $\mathcal{I}_H = \langle L \rangle$ be a background theory, examples and an induction
field, respectively. Let $F$ be a bridge theory wrt $B$ and $E$. A clausal theory $H$ is derived
by inverse subsumption with minimal complements from $F$ wrt $\mathcal{I}_H$ if $H$ is constructed as follows.
\begin{itemize}
\item Step 1. Compute $Taut(\mathcal{I}_H)$;
\item Step 2. Compute $\tau(M(F \cup Taut(\mathcal{I}_H)))$;
\item Step 3. Construct a clausal theory $H$ satisfying the condition:
$H \subsumes \tau(M(F \cup Taut(\mathcal{I}_H)))$.
\end{itemize}
\end{defn}

Inverse subsumption with minimal complements ensures the completeness for finding
hypotheses wrt $\mathcal{I}_H$ and $F$:

\begin{thm}\emph{Completeness of inverse subsumption with minimal complements} Let $B$, $E$ and $\mathcal{I}_H$ be a background theory, examples and an induction field,
respectively. Let $F$ be a bridge theory wrt $B$ and $E$. For every hypothesis $H$ wrt $I_H$ and $F$,
$H$ is derived by inverse subsumption with minimal complements from $F$ wrt $\mathcal{I}_H$.
\end{thm}
\begin{proof}
Follows from \fullref{yamamoto2012inverseLemma2}.
\end{proof}

\section{Bias}

\section{Completeness}

\section{Hypothesis selection}

\section{Hypothesis search}

\section{ILP task definition}

\section{Implementation details}
The yet uncovered details of the Rationale implementation are summarised.

\subsection{Language choice}

\subsection{Libraries used}

\section{Comparison of Rationale with other ILP systems}