\chapter{Rationale ILP system}\label{chap:rationale_ilp_system}
An ILP system Rationale is designed and implemented based on the inverse subsumption with minimal complements algorithm \fullref{inverse_subsumption_with_minimal_complements_algorithm} by Yamamoto\cite{yamamoto2012inverse} and suggestions in \fullref{chap:advancing_solutions_to_issues_in_ilp}. Rationale solves an extension of an ILP task of explanatory induction for clausal theories $B, E^+, E^-, H$. The objective of this chapter is to reveal the details of the prototype implementation and explain the rationales behind.

\section{Algorithm}
Rationale finds all the hypotheses correct wrt to the maximal bridge theory $F=B \land \neg E$ and the induction $\mathcal{I}_H$ by using an antisubsumption on the hypothesis subsumer.


Therefore the steps of the Rationale algorithm are:
\begin{itemize}
\item Step 1. compute the (maximal) bridge theory $F=B \land \neg E$,
\item Step 2. compute the tautologies $Taut(\mathcal{I}_H)$,
\item Step 3. compute the complement $\overline{F \cup Taut(\mathcal{I}_H)}$,
\item Step 4. compute the minimal complement $M(F \cup Taut(\mathcal{I}_H))$,
\item Step 5. remove the tautological clauses to get $\tau(M(F \cup Taut(\mathcal{I}_H)))$.
\item Step 6.  
\end{itemize}

\section{Bias}

\section{Completeness}

\section{Hypothesis selection}

\section{Hypothesis search}

\section{ILP task definition}

\section{Implementation details}
The yet uncovered details of the Rationale implementation are summarised.

\subsection{Language choice}

\subsection{Libraries used}

\section{Comparison of Rationale with other ILP systems}