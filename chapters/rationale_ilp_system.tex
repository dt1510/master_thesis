\chapter{Rationale ILP system}
An ILP system Rationale is designed and implemented based on the work by Yamamoto\cite{yamamoto2012inverse} and suggestions in \fullref{chap:advancing_solutions_to_issues_in_ilp}. Rationale solves an extension of an ILP task of explanatory induction for clausal theories $B, E^+, E^-, H$. The objective of this chapter is to reveal the details of the prototype implementation and explain the rationales behind.

\section{Algorithm outline}
Rationale derives a hypothesis $H$ by inverse subsumption with minimal complements\cite{yamamoto2012inverse} which uses an induction field bias.

\begin{defn}\label{induction_field_definition}\cite{yamamoto2012inverse}
An \emph{induction field}, denoted by $\mathcal{I}_H = \langle L \rangle$,
where $L$ is a finite
set of literals to appear in ground hypotheses.
A ground hypothesis $H_g$ \emph{belongs to} $\mathcal{I}_H$ if
every literal in $H_g$ is included in $L$.
Given an induction field $\mathcal{I}_H = \langle L \rangle$, $Taut(\mathcal{I}_H)$ is defined
as the set of tautologies $\{\neg A \land A | A \in L, \neg A \in L\}$.
\end{defn}



Since we are interested in non-ground hypotheses to be in a hypotheses space we define induction field language as a specific form of a hypothesis bias constructed from the induction field.
\begin{defn}
A hypothesis $H$ is in the \emph{induction field language} $\mathcal{L}(\mathcal{I}_H)$ of an induction field $\mathcal{I}_H=\langle L \rangle$ iff
1) $H$ is a ground hypothesis and $H$ belongs to $\mathcal{I_H}$ or
2) $H$ has a ground instance $H_g$ that belongs to $\mathcal{I_H}$.
\end{defn}

Definition 5 (Hypothesis wrt IH and F ) Let H be a hypothesis. H is a hypothesis wrt
IH and F if there is a ground hypothesis Hg such that Hg consists of instances from H ,
F | ¬Hg and Hg belongs to IH .
Now, the generalization procedure based on inverse subsumption with minimal comple-
ments is as follows:
Definition 6 Let B, E and IH = L be a background theory, examples and an induction
field, respectively. Let F be a bridge theory wrt B and E. A clausal theory H is derived
by inverse subsumption with minimal complements from F wrt IH if H is constructed as
follows.
Step 1. Compute Taut(IH );
Step 2. Compute τ (M(F ∪ Taut(IH )));
Step 3. Construct a clausal theory H satisfying the condition:
H
τ (M(F ∪ Taut(IH ))).


\section{Bias}


\section{Completeness}

\section{Hypothesis selection}

\section{Hypothesis search}

\section{ILP task definition}

\section{Implementation details}
The yet uncovered details of the Rationale implementation are summarised.

\subsection{Language choice}

\subsection{Libraries used}

\section{Comparison of Rationale with other ILP systems}