\chapter{Evaluation}
We evaluate the achievements of the thesis relative to the initial goals and then assess the contribution of its main achievements:
\begin{itemize}
\item formalisation of a language bias and theoretical correspondence between mode declarations, determinations, metaconstraints and production field,
\item experimental classification of ILP systems,
\item proof of completeness of Imparo by inverse subsumption,
\item extension of the results Inverse subsumption for complete explanatory induction and their implementation Rationale ILP system.
\end{itemize}

\section{Evaluation relative to initial goals}
The project timeline is presented which is then subsequently used to reflect on the actual achievements relative to initial goals.

Being interested in artificial intelligence, the initial goal of the author was to conduct a classification of ILP systems as mathematical structures from which an insight could be born into how to design an ILP system that would be able to solve learning problems at a level of human intelligence and greater.


The author outlines the turning points of the project timeline which relate the initial goals to what has been achieved and provide a more accurate assement of the final achievements taking into consideration the difficulties faced:

\begin{itemize}
\item classification of ILP systems as mathematical objects,
\item foundations of ILP,
\item retreat to the experimental classification of ILP systems,
\item work on the theory of ILP,
\item extended classification of ILP systems.
\end{itemize}

The author analyses the turning points and evaluates the contributions towards the initial goal.

\subsection{Classification of ILP systems as mathematical objects}
\subsection{Foundations of ILP}
\subsection{Retreat to the experimental classification of ILP systems}
\subsection{Work on the theory of ILP}
\subsection{Extended classification of ILP systems}

\section{Formalisation of a language bias}
A good description of the language bias can be found in ILP theory and methods by Muggleton and De Raedt \cite{muggleton1994inductive} who list 4 notable frameworks for a language bias specification: the inductive logic programming language of
Bergadano \cite{bergadano1993interactive}, the antecedent description grammars of Cohen \cite{cohen1994grammatically}\cite{cohen1992compiling}, the schemata of
the BLIP-MOBAL team \cite{emde1983discovery}\cite{kietz1992controlling}, their variants \cite{de1992interactive}\cite{silverstein1991relational}
\cite{tausend1994representing} and the fourth framework parametric languages as defined by \cite{muggleton1992efficient}\cite{de1992interactive}
\cite{buntine1987induction}\cite{cohen1993learnability}.
These theoretical frameworks are designed to support a cetain feature of the language bias and its specification: Bergadano's framework aims at readability, the framework by Cohen targets generality and efficiency in computing power.
But most of these specification frameworks have been used only by a small number of ILP systems.

Another language bias specification framework has been developed by Inoue \cite{inoue1992linear} who defined a production field restricting the hypothesis space to the set of relevant clauses called characteristic clauses. However, this specification is implemetation independent and capture speficics of an ILP system.

We provide a specification carrying the information about the specifics of an ILP system while being general enough to be compatible with the language bias specifications of the classified systems.
Our work takes existing implementations of Progol, Aleph, Toplog, Xhail, Imparo, Tal; extracts their specifications of the language bias and designes a system of specification of a language bias by mode declarations, determinations and metaconstraints that is compatible with the specifications of ILP systems classified. An existing formalisation of the mode declarations by Muggleton \cite{muggleton1995inverse} is used. However, a bias specification for determinations \ref{bias_determinations} and metaconstraints \ref{bias_metaconstraints} is new to the author's knowledge. Finally, the validity of a good formalisation of these notions is confirmed by establishing a correspondence \ref{bias_correspondence} with the \emph{implementation idependent} language bias specification of a production field by Inoue \cite{inoue1992linear}. This correspondence can used to reason about a language bias of ILP systems as a production field in a more theoretical way.

Our specification of a language bias is compatible with specifications of all ILP systems using the mode declarations, determinations and metaconstraints. However, there are two problems.

The first being a strong abstraction of the metaconstraints which may encompass almost any form of a language bias specification, but for this reason it does not provide us with much understanding of the underlying specifics of the language specification. Therefore the correspondent classification of ILP systems by metaconstraints is either too general concerned with a bias $\mathcal{H}$ or is too specific concerned with an individual metaconstraint such as the number of clauses in a hypothesis $H$. The future work therefore may include grouping metaconstraints by properties into classes and comparing ILP systems based on whether they support a certain class of metaconstraints.

The second problem is that our formalisation of a langauge bias is specific to only 6 ILP systems classified and extends only to ILP systems with a compatible language bias specification such as Golem \cite{muggleton1992efficient}, CF-induction \cite{yamamoto2014cfInductionWebsite}, but does not extend to other systems like Metagol \cite{muggleton2013meta}\cite{muggleton2014meta}.

\section{Experimental classification}
\section{Proof of Imparo's completeness by inverse subsumption}
\section{Inverse subsumption for complete explanatory induction extensions}