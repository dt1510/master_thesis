\chapter{Evaluation}
Being interested in artificial intelligence, the initial goal of the author was to conduct a classification of ILP systems as mathematical structures from which an insight could be born into how to design an ILP system that would be able to solve learning problems at a level of human intelligence and greater. The author evaluates the achievements of the project wrt to the initial goal and the work conducted by other researchers.

\section{Project timeline}
The author provides a provides turing points in the project from which insights on the critical appraisal of the work can be drawn.

\begin{itemize}
\item classification of ILP systems as mathematical objects,
\item foundations of ILP,
\item retreat to the experimental classification of ILP systems,
\item work on the theory of ILP,
\item extended classification of ILP systems.
\end{itemize}

\section{Achievements wrt to initial goal}
The author apraises the work achieved by the chapters.

\section{Achievements}
The author evaluates the values of the main contributions of the project and their relation to the work conduction
The main contribution of the theis are:
\begin{itemize}
\item formalisation of a language bias and theoretical correspondence between mode \item declarations, determinations, metaconstraints and production field
\item experimental classification of ILP systems
\item proof of completeness of Imparo by inverse subsumption
\item extension of the results Inverse subsumption for complete explanatory induction and their implementation Rationale ILP system
\end{itemize}