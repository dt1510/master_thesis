\chapter{Evaluation}
We evaluate the achievements of the thesis relative to the initial goals and then assess the contribution of its main achievements:
\begin{itemize}
\item formalisation of a language bias and theoretical correspondence between mode declarations, determinations, metaconstraints and production field,
\item experimental classification of ILP systems,
\item proof of completeness of Imparo by inverse subsumption,
\item extension of the results Inverse subsumption for complete explanatory induction and their implementation Rationale ILP system.
\end{itemize}

\section{Evaluation relative to initial goals}
The project timeline is presented which is then subsequently used to reflect on the actual achievements relative to initial goals.

Being interested in artificial intelligence, the initial goal of the author was to conduct a classification of ILP systems as mathematical structures from which an insight could be born into how to design an ILP system that would be able to solve learning problems at a level of human intelligence and greater.


The author outlines the turning points of the project timeline which relate the initial goals to what has been achieved and provide a more accurate assement of the final achievements taking into consideration the difficulties faced:

\begin{itemize}
\item classification of ILP systems as mathematical objects,
\item foundations of ILP,
\item retreat to the experimental classification of ILP systems,
\item work on the theory of ILP,
\item extended classification of ILP systems.
\end{itemize}

The author analyses the turning points and evaluates the contributions towards the initial goal.

\subsection{Classification of ILP systems as mathematical objects}
\subsection{Foundations of ILP}
\subsection{Retreat to the experimental classification of ILP systems}
\subsection{Work on the theory of ILP}
\subsection{Extended classification of ILP systems}

\section{Formalisation of a language bias}
A language bias in ILP has been intuitively described in 
\section{Experimental classification}
\section{Proof of Imparo's completeness by inverse subsumption}
\section{Inverse subsumption for complete explanatory induction extensions}