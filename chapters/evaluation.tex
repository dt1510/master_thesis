\chapter{Evaluation}

\section{Self-critique}
It is my opinion that the greatest problems in the field of ILP are of foundational nature. However, unless I can demonstrate more insight with the new way to compare ILP systems, there is a strong belief that my opinions are biased despite my search of understanding.
\section{Measuring the success}

\subsection{Work nature}
Since I am approaching the field from the theoretical perspective, measuring the success may not be easy. The results established in theory are often not undestood immediately when there is search of their application.

On the other hand, once a mathematical property about a well-defined mathematical object is proved and the proof is sound, nobody cannot argue about its correctness.

Good definitions and abstractions will reflect in an elegant reasoning about the ILP systems and their properties; and make their comparison simple and intuitive. The problematic definitions will reflect in a difficulty and unusability of the theoretical foundations.

\subsection{Work relevance}
In my opinion my work is important as the need of the classification of ILP systems has been addressed by several ILP researchers. Notably, Muggleton et al in their paper:

MC-TopLog: Complete Multi-clause Learning Guided by a Top Theory - a simple comparison of the single and multi clause systems has been made in order to argue that a new implemented MC-Toplog system was superior to other systems such as Toplog.

With a universaly accepted theory of comparison of ILP systems, ILP researches could focus more on improving definable properties of the ILP systems and thus a clearer direction in ILP research would be set.

\subsection{Size of contribution}
As a MEng student with a limited knowledge, I expect to only give hints on what mathematical foundations of an ILP field may look like, how we may comapare ILP systems, examine their properties.

A successful amount of the work should be evident from the results such as an establishment of a completeness/incompleteness of some ILP method, proving that a hypotheses space of one ILP system is greater than a hypotheses space of another ILP system. A major contribution would represent disproving a statement that ILP researchers intuitively thought to be true.

\subsection{Work impact}
The impact of the work will likely be measurable only after the thesis completion. If thesis serves as an inspiration for ILP researchers to approach the problems in the ILP field with more rigour, then a success is obvious. Particularly, if they start applying certain methods and formalisms developed to compare their new ILP systems and if they identify their priorities of research based on one of such formalisms.
